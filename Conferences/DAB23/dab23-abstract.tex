\documentclass[a4paper]{article}
%\usepackage{simplemargins}

%\usepackage[square]{natbib}
\usepackage{amsmath}
\usepackage{amsfonts}
\usepackage{amssymb}
\usepackage{graphicx}

\usepackage[authordate, backend=biber]{biblatex-chicago}
\addbibresource{/Users/zackbatist/Dropbox/zotero/zack.bib}

\usepackage{parskip}
\setlength{\parindent}{20pt}
\usepackage{geometry}
\geometry{left=1in, right=1in, top=1in,}

\begin{document}
\pagenumbering{gobble}

\large
\begin{center}
Collaborating from a distance: accounting for missing context while using open data platforms

% Author names and affiliations
\large
Zachary Batist\\

\small  
Faculty of Information, University of Toronto\\
z.batist@mail.utoronto.ca\\

\end{center}

\normalsize

Open data platforms have emerged as important elements of global information infrastructures, which are designed to enable digital archaeological research to flourish, especially research that involves integrating and reusing published datasets. While these systems are technically impressive, numerous studies have demonstrated that they fail to communicate important contextual information regarding the underlying premises and circumstances that contribute to data's meaning, and which analysts rely upon to inform their work in secondary research contexts (cf. \cite{faniel2013}; \cite{opitz2021}; \cite{huggett2022}). The common practices of data reusers reaching out to a dataset's originators to obtain this contextual information, and for data producers to produce all their analytical findings "in house" lest the data be applied in ways that they consider to be innapropriate, indicate that archaeologists are not satisfied with this novel arrangement. Instead, archaeologists seek out and maintain collaborative commitments that more closely resemble a discursive connection shared among colleagues, rather than behaving merely as uploaders and downloaders.

This paper, drawn from my doctoral dissertation, examines how archaeologists establish communal data streams within relatively bounded project systems, namely within two individual archaeological projects and within a regional data sharing consortium with limited scope and targeted ressearch objectives. More specifically, I identify social and technological factors that make it feasible to collectively construct archaeological knowledge, and which are lacking in global infrastructures that share a similar goal.

Based on observations and interviews with archaeologists as they work, focused particularly on various documentation practices, I found that records of situated experiences, though often either intangible, ephemeral, or otherwise relatively unamenable to widepsread distribution at scale, are nonetheless crucial for obtaining a critical understanding of formal records' potential value and limitations.

I found that these records end up in their own separate data stream, and serve different purposes than formal records. Data are not just one thing, they may hold different roles. In these cases, they are either memnonic, explanatory, supportive, or auditing devices.

Moreover, a sense of trust is involved in creating and sharing these records. The stuff about sharing field notes and internal records outside a project.

Even writing a report, or publishing an aritcle, involves a certain degree of contextualization that is not present in the data in itself.

These media thus play alternative and supportive roles, and may in fact stand in for collaborative relationships. How might we tie these in to a project repository, in a way that facilitates meaningful communication?

And these documents also have their own context. They are backed by stories or reasons for existing in the state that they are. What kinds of revisions were needed to make a paper ready to publish? Were there prior attempts to analyze material, and re-analysis for new personnel, and if so, why, and how did they differ? These stories often leave no trace in the formal academic literature, and must be retold among trusted colleagues. Full transparency is a lie, the commons always have boundaries and restrictions, and are built with certain stakeholders' priorities in mind.

Perhaps we can leverage these facts of life in our plans to develop information infrastructures. Perhaps we can not. But then how can we supplement or account for these gaps?

This is where a need for change in praxis comes into play. We need to change the conditions of the game. Projects led by a single individual, while useful, bring with them their own issues, and attempts to re-hash this arrangement (i.e. catalhoyuk) have arguably failed due to their inability to alter the broader conditions under which archaeology exists.




\hspace{10pt}

\normalsize
\noindent
Keywords: collaboration, communication, data governance\\
Type: oral presentation

\printbibliography

\end{document}