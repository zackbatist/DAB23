\documentclass[a4paper]{article}
%\usepackage{simplemargins}

%\usepackage[square]{natbib}
\usepackage{amsmath}
\usepackage{amsfonts}
\usepackage{amssymb}
\usepackage{graphicx}

\usepackage[authordate, backend=biber]{biblatex-chicago}
\addbibresource{/Users/zackbatist/Dropbox/zotero/zack.bib}

\usepackage{parskip}
\setlength{\parindent}{20pt}
\usepackage{geometry}
\geometry{left=1in, right=1in, top=1in,}

\begin{document}
\pagenumbering{gobble}

\large
\begin{center}
Documenting the collaborative commitments that support data sharing within archaeological project collectives

% Author names and affiliations
\large
Zachary Batist\\

\small  
Faculty of Information, University of Toronto\\
z.batist@mail.utoronto.ca\\

\end{center}

\normalsize

Archaeological research is inherently collaborative, in that it involves many people coming together to examine a material assemblage of mutual interest by implementing a variety of tools and methods in tandem. Independent projects establish organizational structures and information systems to help coordinate labour and pool information derived thereof into a communal data stream, which can then be applied towards the production and publication of analytical findings. Albeit not necessarily egalitarian, and with different expectations set for people assigned different roles, archaeological projects thus constitute a form of commons, whereby participants contribute to and obtain value from a collective endeavour. Adopting open research practices, including sharing data beyond a project's original scope, involves altering the collaborative commitments that bind work together. This paper, drawn from my doctoral dissertation, examines how archaeologists are presently navigating this juncture between established professional norms and expectations on the one hand, and the potential benefits and limitations afforded by open research on the other.

I applied an abductive qualitative data analysis approach based on recorded observations, interviews, and documents collected from three cases, including two independent archaeological projects and one regional data sharing consortium with limited scope and targeted research objectives. My analysis documents a few underappreciated aspects of archaeological projects' sociotechnical arrangements that open data infrastructures should account for more thoroughly:

\begin{enumerate}

\item boundaries, whether they restrict membership within a collective, delimit a project's scope, or limit the time frame under which a project operates, have practical positive value, and are not just arbitrary impediments;

\item systems designed to direct the flow of information do so via the coordination of labour, and the strategic arrangement of human and object agency, as well as resistances against such managerial control, are rarely accounted for in data documentation; and

\item information systems and the institutional structures that support them tend to reinforce and reify existing power structures and divisions of labour, including implicit rules that govern ownership and control over research materials and that designate who may benefit from their use.

\end{enumerate}

By framing data sharing, whether it occurs between close colleagues or as mediated by open data platforms among strangers, as comprising a series of collaborative commitments, my work highlights the broader social contexts within which we develop open archaeological research infrastructures. As we move forward, we should be aware of and account for how the data governance models embedded within open research infrastructures either complement or challenge existing social dynamics.

\hspace{10pt}

\normalsize
\noindent
Keywords: collaboration, social context, data governance\\
Type: oral presentation

\printbibliography

\end{document}