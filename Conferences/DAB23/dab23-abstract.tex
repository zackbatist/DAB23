\documentclass[a4paper]{article}
%\usepackage{simplemargins}

%\usepackage[square]{natbib}
\usepackage{amsmath}
\usepackage{amsfonts}
\usepackage{amssymb}
\usepackage{graphicx}

\usepackage[authordate, backend=biber]{biblatex-chicago}
\addbibresource{/Users/zackbatist/Dropbox/zotero/zack.bib}

\usepackage{parskip}
\setlength{\parindent}{20pt}
\usepackage{geometry}
\geometry{left=1in, right=1in, top=1in,}

\begin{document}
\pagenumbering{gobble}

\large
\begin{center}
Collaborating from a distance: accounting for missing context while using open data platforms

% Author names and affiliations
\large
Zachary Batist\\

\small  
Faculty of Information, University of Toronto\\
z.batist@mail.utoronto.ca\\

\end{center}

\normalsize

Archaeological research is inherently collaborative, in that it involves many people coming together to examine a material assemblage of mutual interest through the use of various tools and methods. Within individual projects, this involves establishing organizational structures and information systems to help coordinate labour and pool information derived thereof into a communal data stream, which can then be applied towards productive ends. Albeit not necessarily egalitarian, and with different expectations set for people assigned different roles, archaeological projects thus constitue a form of commons, whereby participants contribute to and obtain value from a collective endeavour. Open research systems are also a form of commons, and hold their own distinctive roles, boundaries, governance models, and collaborative commitments that archaeologists are still coming to grips with.

This paper examines how archaeologists are engaging with and creatively pursuing novel sets of collaborative commitments at the intersections between established professional norms and expectations on the one hand, and the potential benefits and limitations afforded by open research infrastructure on the other. More specifically, I examine data sharing practices within relatively bounded project systems, namely within two individual archaeological projects and within a regional data sharing consortium with limited scope and targeted ressearch objectives. By framing data sharing, whether it occurs between close colleagues or as mediated by open data platforms among strangers, as comprising a series of collaborative commitments, we can develop open research systems that are tailored to archaeology's distinct epistemic culture, rather than pushing entirely new ways of working with others.

I identify a few particularly salient social factors that play a large role in coordinating archaeological labour and data, and which open data infrastructures tend to ignore: boundaries, whether they restrict membership within a collective, delimit a project's scope, or limit the time frame under which a project operates, have practical positive value; the systems that are designed to direct the flow of information do so by coordinating labour, and data management should thus be treated as a form of management writ large, which involves strategic arrangement of human and object agency, as well as resistance against such means of control; and the fact that information systems and the institutional structures that they are built upon tend to reinforce and reify existing power structures and divisions of labour, including implicit rules that govern ownership and control over research materials and data.

Participating in open data is a social experience, and thus involves complex power dynamics and acts of exclusion. The main concern should be how we navigate these challenges, and how we develop data governance strategies that complement or challenge existing norms.




\hspace{10pt}

\normalsize
\noindent
Keywords: collaboration, communication, data governance\\
Type: oral presentation

\printbibliography

\end{document}