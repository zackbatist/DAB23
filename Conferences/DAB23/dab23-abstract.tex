\documentclass[a4paper]{article}
%\usepackage{simplemargins}

%\usepackage[square]{natbib}
\usepackage{amsmath}
\usepackage{amsfonts}
\usepackage{amssymb}
\usepackage{graphicx}

\usepackage[authordate, backend=biber]{biblatex-chicago}
\addbibresource{/Users/zackbatist/Dropbox/zotero/zack.bib}

\usepackage{parskip}
\setlength{\parindent}{20pt}
\usepackage{geometry}
\geometry{left=1in, right=1in, top=1in,}

\begin{document}
\pagenumbering{gobble}

\large
\begin{center}
Collaborating from a distance: accounting for missing context while using open data platforms

% Author names and affiliations
\large
Zachary Batist\\

\small  
Faculty of Information, University of Toronto\\
z.batist@mail.utoronto.ca\\

\end{center}

\normalsize

Open data platforms have emerged as important elements of global information infrastructures, which are designed to enable digital archaeological research to flourish, especially research that involves integrating and reusing published datasets. While these systems are technically impressive, numerous studies have demonstrated that they fail to broker a meaningful sense of trust between data producers and reusers. For instance, \cite{faniel2013} reported that data reusers seek information about datasets' originators to gauge their potential value; \cite{opitz2021} highlighted scholars' apprehensions when relinquishing control over how their data will be used in alternative research contexts; and \cite{huggett2022} raised significant concerns regarding the epistemic distance between contexts of data creation and data reuse that is afforded by these technological platforms. These problems demonstrate a collectively-held sense that there is more to a dataset than what is captured in records and documentation shared on open data platforms. Interestingly, archaeologists respond to these challenges by seeking ways to relate on a subtextual level, in what might most effectively entail a traditional collaborative relationship. It is quite ironic, then, that the formal and transactional protocols meant to streamline mutual comprehension of a dataset reveal their own inadequacy for achieving their stated purpose and the strengths of the system that they are meant to replace.

This paper examines this issue by highlighting factors that contribute to data sharing in relatively bounded collaborative systems, namely within two individual archaeological projects and within a regional data sharing consortium with limited scope and targeted ressearch objectives. More specifically, I examine how collaborating project participants establish mutual comprehension and trust, by comparing the creation and management of formal records with various means of communicating situated experiences, including the use of field journals, site tours, published reports, newsletters, and other vectors of personal communication. My findings hightlight that situated experiences, though crucial for obtaining a critical understanding of formal records' potental value and limitations, form their own, less computationally useful sets of representative media, which are not as amenable to widespread distribution at scale. I then consider how we might account for this contextual information in our global open information infrastructures and through alternative data governance models.

\hspace{10pt}

\normalsize
\noindent
Keywords: collaboration, communication, data governance\\
Type: oral presentation

\printbibliography

\end{document}