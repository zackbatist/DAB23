\documentclass[a4paper]{article}
%\usepackage{simplemargins}

%\usepackage[square]{natbib}
\usepackage{amsmath}
\usepackage{amsfonts}
\usepackage{amssymb}
\usepackage{graphicx}

\usepackage[authordate, backend=biber]{biblatex-chicago}
\addbibresource{/Users/zackbatist/Dropbox/zotero/zack.bib}

\usepackage{parskip}
\setlength{\parindent}{20pt}
\usepackage{geometry}
\geometry{left=1in, right=1in, top=1in,}

\begin{document}
\pagenumbering{gobble}

\large
\begin{center}
Collaborating from a distance: accounting for missing context while using open data platforms

% Author names and affiliations
\large
Zachary Batist\\

\small  
Faculty of Information, University of Toronto\\
z.batist@mail.utoronto.ca\\

\end{center}

\normalsize

Open data platforms have emerged as important elements of global information infrastructures, which are designed to enable digital archaeological research to flourish, especially research that involves integrating and reusing published datasets. While these systems are technically impressive, numerous studies have demonstrated that they fail to communicate important contextual information regarding the underlying premises and circumstances that contribute to data's meaning, and which analysts rely upon to inform their work in secondary research contexts (cf. \cite{faniel2013}; \cite{opitz2021}; \cite{huggett2022}). The common practices of data reusers reaching out to a dataset's originators to obtain this contextual information, and for data producers to produce all their analytical findings "in house" lest the data be applied in ways that they consider to be innapropriate, indicate that archaeologists are not satisfied with this novel arrangement. Instead, archaeologists seek out and maintain collaborative commitments that more closely resemble a discursive connection shared among colleagues, rather than behaving merely as uploaders and downloaders.

This paper, drawn from my doctoral dissertation, examines how archaeologists establish communal data streams within relatively bounded project systems, namely within two individual archaeological projects and within a regional data sharing consortium with limited scope and targeted ressearch objectives. More specifically, I identify social and technological factors that make it feasible to collectively construct archaeological knowledge.

My findings indicate that records of situated experiences, though often either intangible, ephemeral, or otherwise relatively unamenable to widepsread distribution at scale, are nonetheless crucial for obtaining a critical understanding of formal records' potential value and limitations.

I also consider how projects' organizational structures and information systems direct the flow of archaeological labour and data derived thereof towards productive ends; data governance models, which include implicit professional expectations regarding who may access and benefit from the collectively-maintained pool of information generated through years of distributed labour, are significant factors that effect how collaboration occurs within and beyond a project's scope.



I then consider how we might account for this contextual information in our global open information infrastructures and through alternative data governance models.

which may inform the development of information systems that operate at a global scale.


\hspace{10pt}

\normalsize
\noindent
Keywords: collaboration, communication, data governance\\
Type: oral presentation

\printbibliography

\end{document}