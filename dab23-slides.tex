\documentclass{beamer}
\usepackage[
type={CC},
modifier={by-sa},
version={4.0},
]{doclicense}
\setbeamersize{text margin left=5mm,text margin right=5mm} 
\setbeameroption{show notes}
\setbeamertemplate{note page}{%
    % adds in bg color
        \insertvrule{\paperheight}{note page.bg}%
        \vskip-\paperheight%
    % changes font color
        \usebeamercolor[fg]{note page}%
    % inserts note
        \insertnote%
}
\AtBeginNote{\hspace*{-20pt}\addtolength\leftmargini{-20pt}}
\AtEndNote{\addtolength\leftmargini{20pt}}
\usetheme{metropolis}
\title{Documenting the collaborative commitments that support data sharing within archaeological project collectives}
\date{\today}
\author{Zachary Batist}
\institute{Faculty of Information\\University of Toronto}

\begin{document}

\begin{frame}
  \maketitle

  \note[item]{Hi everyone, glad to be here.}
  \note[item]{The paper I'm about to present draws from my dissertation, which ins't strictly about open science but is heavily inspired by challenges that digital archaeologists commonly face while engaging with open research data,}
  \note[item]{namely the awkwardness of opening a spreadsheet and entering a world of tacit assumptions and experiences that differ from your own, and the actions taken to reconcile these disparities.}


\end{frame}


\begin{frame}{Objectives}
  \begin{enumerate}
    \item Articulate the collaborative commitments and technological systems that mediate data sharing on a local, collegial level
    \item Relate these accepted norms and expectations to the alternative sociotechnical arrangements that frame participation in open data infrastructures
    
    \note[item]{I consider uploading and downloading formally arranged information from a digital repository as just one form of data sharing among many other possibilities.}
    \note[item]{Archaeologists share data all the time, following numerous protocols that don't necessarily involve making the records available 24/7 on the web.}
    \note[item]{My work basically tries to articulate all the different kinds of data sharing that archaeologists do, while calling attention to the collaborative commitments that they entail and the role of mediating technologies in these interactions.}
    % \note[item]{For instance, I examine how archaeologists within projects stake ownership over a dataset, which may be expressed in terms of the amount of recognized effort put into the construction of a dataset, associations with a tool used to produce it, one's status within the project hierarchy, or an explicit mandate provided by an authoritative figure or institution.}
    \note[item]{I think that many of the problems that archaeological information infrastructures have to deal with relate to an inability to grapple with archaeologists' discomfort and apprehension regarding the expansion if their collaborative commitments into new dimensions for which we have not yet established social and professional norms.}
    \note[item]{So I'm here to shed some light on some potential incongruities between how archaeologists tend to share data on a local and collegial scale, and the very different sociotechnical arrangements that tend to be imposed by open data infrastructures, at least under the paradigm we have opted to pursue.}

  \end{enumerate}
  
\end{frame}

\begin{frame}{My Approach}

  \begin{itemize}
    \item I examine archaeological research as cultural practice
    \item Archaeology is an inherently collaborative and social enterprise
    \item Archaeology necessarily relies on social and technological systems to mediate and encourage cooperative and productive behaviour
    \item Open data infrastructures rearrange these sociotechnical systems in potentially drastic ways (which is not necessarily a good thing)
  \end{itemize}

  \note[item]{First a bit about my approach.}
  \note[item]{I examine archaeological research as a cultural practice, as work that occurs within a social environment.}
  \note[item]{I'm less interested in what archaeologists are looking at than how they do it, and crucially, how the meanings they ascertain about things are validated through participation within a collective enterprise.}
  \note[item]{So I set out to observe and interview archaeologists as they work, and articulate the social and technological circumstances that underlie the formation of a communal data stream.}
\end{frame}
  

  \begin{frame}{Data Collection and Analysis}
    \begin{columns}[T,onlytextwidth]

      \column{0.45\textwidth}
      \begin{itemize}
        \item Data collected over four years, from 2017 - 2020
        \item On site, museum, dig house, and office settings, both on and off season
        \item Abductive qualitative data analysis methods roughly based on grounded theory
      \end{itemize}

      \column{0.5\textwidth}
    \centering
    \includegraphics[width=4cm,]{images/fig1.png}
  \end{columns}

  \note[item]{I recorded over 90 hours of archaeologists doing their work, interviewed over 21 participants, and examined the documents that archaeologists produce and handle on a regular basis.}
  \note[item]{I did this over several years, and three cases are represented in my work.}
  \note[item]{Observations and interviews were held in settings where people were excavating, collecting samples, processing and identifying finds, working at their laptops, giving site tours, and holding casual conversations.}
  \note[item]{All of the names presented here are pseudonyms, by the way.}
  \note[item]{I then applied qualitative data analysis methods derived from grounded theory to sort through my vast dataset and articulate some trends and theories that underlie them, some of which are presented here.}

\end{frame}

% \begin{frame}{Archaeology is inherently collaborative}
%   \begin{itemize}
%     \item Work is distributed across members of project teams
    % \item Activities are performed asynchronously and in tandem across time and place
%     \item Projects draw from and extend upon prior and concurring work
    % \item Professional standards and community expectations dictate how work should be performed and what derived outcomes should look like
%   \end{itemize}
% 
% 
  % \note[item]{I see archaeology as inherently collaborative, involving careful coordination of work across time, place, and social context.}
  % \note[item]{Moreover, projects draw from and extend from prior and concurring work.}
  % \note[item]{All of these interactions are governed by professional standards and community expectations.}
  % \note[item]{For instance, you don't simply arrive at a site and ask to dig, just as you wouldn't suddenly dance naked in the street and expect others not to be concerned.}
% 
% \end{frame}
% 
% \begin{frame}{Archaeologists produce media to facilitate collaboration}
%   \begin{itemize}
%     \item Media are externalized representations of ideas or experiences
    % \item Include both tangible and computationally useful records, and more malleable means of communication
%   \end{itemize}
%   
%   \metroset{block=fill}
%   \begin{columns}[T,onlytextwidth]
%     \column{0.45\textwidth}
%       \begin{block}{Formal Records}
%         \begin{itemize}
%           \item Recording sheets
%           \item Spreadsheets
%           \item Stratigraphic sequences
%         \end{itemize}
%       \end{block}
% 
%     \column{0.4\textwidth}
%       \begin{block}{Informal Records}
%         \begin{itemize}
%           \item Email, phone calls
%           \item Field manuals
%           \item Stories
%         \end{itemize}
%       \end{block}
%   \end{columns}
% 
%   \begin{itemize}
    % \item All enable ideas and personal experiences to be shared, accumulated, and rendered useful
    % \item Different kinds of records tend to be used and valued in different ways
%   \end{itemize}
%   
  % \note[item]{As externalized representations of ideas or experiences, media play a particularly significant role in facilitating collaborative ties among archaeologists.}
  % \note[item]{Media can include things like recording sheets, spreadsheets and trench reports, emails, field manuals, and stories.}
  % \note[item]{All of these things enable ideas and personal experiences to be shared, accumulated, and rendered useful within a collective work environment.}
  % \note[options]{Moreover, handling various kinds of media is circumscribed by social and professional norms.}
  % \note[item]{For instance, excavators are encouraged to limit their personal voice when filling in recording sheets and trench reports, but may recall their personal experiences through stories and other media.}
% 
% \end{frame}

% \section{Methods and Data}
% 
% \begin{frame}{Data Collection}
  % \metroset{block=fill}
  % \begin{columns}[T,onlytextwidth]
    % \column{0.52\textwidth}
    % \begin{block}{Observation}
      % \begin{itemize}
        % \item Video, audio, handwritten notes
        % \item 90+ hours of work observed
        % \item 150+ hours of recorded footage
      % \end{itemize}
    % \end{block}
% 
    % \begin{block}{Document Analysis}
      % \begin{itemize}
        % \item Recording sheets
        % \item Photos
        % \item Labels and tags
        % \item Database entries
        % \item Spreadsheets
        % \item Trench reports ...
      % \end{itemize}
    % \end{block}
% 
    % \column{0.45\textwidth}
% 
    % \begin{block}{Retrospective Interviews}
      % 31 interviews held with 21 distinct participants about:
      % \begin{itemize}
        % \item the project’s goals
        % \item publication strategies
        % \item collaboration
        % \item teaching and learning ...
      % \end{itemize}
    % \end{block}
% 
    % \begin{block}{Contextual Interviews}
      % Explanations and reflections about specific tasks as work proceeded
    % \end{block}
% 
  % \end{columns}
% \end{frame}
% 
% % \begin{frame}{Data Collection}
  % % \begin{itemize}
    % % \item Observational records (video, audio, handwritten notes)
    % % \begin{itemize}
      % % \item 90 + hours of work observed
      % % \item 150 + hours of recorded footage
    % % \end{itemize}
    % % \item Episodic interviews
    % % \begin{itemize}
      % % \item Explanations, think-aloud, as work proceeded
    % % \end{itemize}
    % % \item Retrospective interviews
    % % \begin{itemize}
      % % \item Thoughts about the project’s goals, publication strategies,
      % % \item Collaboration, teaching and learning, etc
      % % \item 31 interviews held with 21 distinct participants
    % % \end{itemize}
    % % \item Document analysis
    % % \begin{itemize}
      % \item Forms, photos, labels, DB entries, spreadsheets, trench reports, % etc
    % % \end{itemize}
  % % \end{itemize}
% % \end{frame}
% 
% 
% \begin{frame}{Data Collection}
  % \begin{columns}
    % \column{0.55\textwidth}
      % \begin{itemize}
        % \item Data collected over four years, from 2017 - 2020
        % \item Observations conducted on site, in museum, and in dig house settings
        % \item Interviews held during summer fieldwork and during the off-season
        % \item Abductive qualitative data analysis methods using MaxQDA
        % \begin{itemize}
          % \item Following Charmaz 2014, \textit{Constructing Grounded Theory}
        % \end{itemize}
        % \item Names of people and places are pseudonyms
      % \end{itemize}
% 
      % \column{0.4\textwidth}
      % \centering
      % \includegraphics[width=2cm,]{Picture2.png}
        % \includegraphics[width=5cm]{Picture1.png}
% 
% \end{columns}
% \end{frame}


% \begin{frame}{Cases}
  % Case 1
  % \begin{itemize}
    % \item Longitudinal (2017-2019), carte blanche access to all their records
    % \item Prehistoric site in southern Europe, follows fairly traditional methods and data management practices
  % \end{itemize}
% 
  % Case 2
  % \begin{itemize}
    % \item Hellenistic site in southern Europe
    % \item Very tech-savvy, early adopter of digital tools and methods
    % \item Participation spanned one week towards the end of fieldwork in 2019
  % \end{itemize}
% 
  % Case 3
  % \begin{itemize}
    % \item A data sharing consortium focused on a specific region
    % \item Focused on technical and social challenges re: data sharing
    % \item Entirely based on interviews and handwritten notes from 2019-2020
  % \end{itemize}
% \end{frame}

\begin{frame}{Archaeological Projects}
  

  \note[options]{The first thing I want to highlight is the role of professional boundaries in the creation and use of archaeological data.}
  \note[options]{One kind of boundary that is particular important is the boundary around a project.}
  \note[options]{A project is granted a license to exclusively examine an archaeological assemblage typically spatial bounded by what we call a site.}
  \note[options]{Attached to the license is a lot of money, which can be leveraged to produce prestige and advance one's careers and the careers of friends and colleagues}
  \note[options]{A project director who applied for the license and funding typically has universal control over how that funding is allocated, and is responsible for inviting others to participate in the project.}
  \note[options]{In effect, he is therefore providing opportunities for people to have access to a scarce resource, namely archaeological finds, and to transform them into legitimate records about the past.}
  \note[options]{These records, which are manifested as published articles, are the currency through which someone builds their career.}
  \note[options]{Nice little racket, eh?}
  \note[options]{Anyway, the control held by the director generally corresponds with the scope of the project as a whole.}
  \note[options]{Just as with corporate structures, the domain of the firm is bounded by what is owned by the owners.}
  \note[options]{The database is also controlled by the director, but we will get to that in a moment.}
\end{frame}

\begin{frame}{Divisions of Labour (based on what they work with)}

  \note[options]{Directors allocate responsibilities over certain material assemblages to specialists and supervisors.}
  \note[options]{These delegated individuals take their experiences with the materials, transform them into written documents, and pass them along to the director in the form of a report.}
  \note[options]{They may also share more granular, formally arranged information for tight integration into a centralized database, which can then be retrieved along other relevant data.}
  \note[options]{In general, however, published articles tend to lack much of this potential for integration, with results being presented on a subdisciplinary basis in turn.}
  \note[options]{That does not mean that there is no collaboration.}
  \note[options]{I observed numerous excited conversations between geoarchaeological specialists of various sorts, who plan and strategize where to excavate and collect samples from in a proactive way.}
  \note[options]{However, this does not translate to a publishable document.}
  \note[options]{The division of labour that corresponds with material assemblages is only one axis through which labour is divided, and it is in fact an axis that is only useful for allocating who owns what assemblages and data.}

\end{frame}

\begin{frame}{Divisions of Labour (based on agency)}
  \note[options]{This is more about one's ability to participate in a conversation and have the potential to influence the way others work.}
  \note[options]{In other words, another division of labour is characterized by different kinds of agency.}

  People in different roles hold different amount of creative agency
  They also hold different capability to participate within broader discourse (matroshka doll)
  The boundaries between human and tool agency get very blurred, as people are managed and leveraged to produce targeted outputs
  
\end{frame}

\begin{frame}{Agency and activity systems}
  \note[options]{Note that all of this is based in context.}
  \note[options]{A director does not necessarily have creative agency in the experience of a trench. Instead, they hold supportive agency, providing the tools and resources necessary to facilitate good work performed by creative and automated actors. Once the information collected by the trench leaves that setting and enters the dig house or the database, it enters a different functional realm, no longer a product derived from intimate familiarity with a trench, and becoming a generic resource to be leveraged by a publishing member of the team who relies on analyst and computational work environment as supportive agents, who provides him with findings to be put in a narrative presented by the director.}
  \note[options]{It is noteworthy that only the activity contexts where upper management wields creative agency hold the potential to garner some legitimately valuable outcome. Excavators produce records that are not useful in themselves. Even field journals, which are actually much richer and informative than other media produced elsewhere in any given project, are left out of the centralized database and generally ignored and not taken seriously (unless they were written by some long dead white man from the colonial era, such as Flinders Petrie and Arthur Evans).}
\end{frame}

\begin{frame}{Snippets}
  Quote from Case 1 about managing conflict, ensuring no one is stepping on each others' shoes
  Quote from Case 3 about maintaining autonomy
  Quote from Case 3 about the differences between projects
  Quote from Case 2 about the director being the primary actor, and compare with other collaborative systems like film direction and corporate ownership whereby the scope of the company is dictated by what is and is not owned by the owners
  
\end{frame}

\begin{frame}{Information flows}
  Projects are sociotechnical systems, assembled to produce a central record pertaining to a spatially or topically bounded material assemblage
  Projects try to impose a sense of order, to dictate the flow of information
  This is through organizational structures like dictating who responds to who, and through technical protocols

  Different kinds of information are valued in different ways
  The things included in an information system are geared towards producing certain kinds of outputs
  Things excluded from the information system are notably things that do not get you to those desired kinds of outputs
  Are there patterns in who creates or works with these things?
  
\end{frame}

\begin{frame}{Snippets}
  Figures depicting a typical project hierarchy, and how this maps onto the material assemblage; each specialist maintains control over their own material domain
  Highlight the role of reports in transmitting information from specialists to centralizing systems controlled by directors
  Quote from Case 1 about the role and value of trench notebooks
  Notes from Case 1 about the role and value of site tours
  Quote from Case 1 about the need to convince a visiting specialist, and the transformation of this experience into a "personal communication" citation and into word of mouth acceptance of findings
  Quotes from Case 3 on extraneous and potentially embarassing information included in informal records
  

\end{frame}


\begin{frame}{Information Management as Management}
  The systems that manage data do so by managing the labour that produces data

\end{frame}

\begin{frame}
  Cases 1, 2 and 3 on the directors ownership and control over the datasets they govern

\end{frame}

\section{Tensions with Open Data Infrastructures}

\begin{frame}{XXXX}
  Projects remain the fundamental units of concern, those boundaries are maintained
  Power is further concentrated in the directors' position, as the primary contact, and as the recognized owner and controller of the totality of a project's information; the boundaries within a project tend to be rendered invisible
  Datasets are presented as they are in their final state; there is essentially no mention of the mechanisms that governed data flow (cite Isto Huvila and his colleagues); perhaps this is simply a matter of general sense of mutual understanding, a product of institutional isomorphism and immersion within the community of practice (something about seeming completely clueless if you include computational specifications at a certain degree of detail)

\end{frame}

\begin{frame}{Where do we go from here?}
  Goal-oriented data sharing
  Report-oriented publication strategy; assemblage of specialist reports, rather than a single cohesive document
  Provide contact information
  Registry rather than repository mentality
  Accept and acknowledge the complicated imperfections, like fuzzy notions of authorship and spotty inclusion of informal records in digital repositories; imposing structures and standards is not the solution, we should instead treat these as informal aspects of our own cultural practice, something to be talked about freely and reiteratively discussed and reflected upon; there is no need for a formal solution to give artificial and unwanted structure, embrace the anarchy


\end{frame}

\section*{Back matter}

\begin{frame}[standout]
  Questions?
\end{frame}

\begin{frame}
  
\begin{center}You can download this presentation at\\
\url{github.com/zackbatist/DAB23}\end{center}

\begin{center}\doclicenseImage[imagewidth=2cm]\\
\footnotesize \href{http://creativecommons.org/licenses/by-sa/4.0/}{Creative Commons Attribution-ShareAlike 4.0}\end{center}


% \begin{center}\ccbysa\end{center}

\end{frame}

\begin{frame}{References}
  \bibliography{demo}
  \bibliographystyle{abbrv}
\end{frame}

\end{document}