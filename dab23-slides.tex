\documentclass{beamer}
\usepackage[
type={CC},
modifier={by-sa},
version={4.0},
]{doclicense}
\setbeamersize{text margin left=5mm,text margin right=5mm} 
\usetheme{metropolis}

\setbeamercolor{block body}{bg=mDarkTeal!30}
\setbeamercolor{block title}{bg=mDarkTeal,fg=black!2}

\title{Documenting the collaborative commitments that support data sharing within archaeological project collectives}
\author{Zachary Batist}
\date{Digital Archaeology Bern 2023}
\institute{Faculty of Information \\ University of Toronto}

\begin{document}
\begin{frame}
  \maketitle
\end{frame}

\begin{frame}{Objectives}
  \begin{enumerate}
    \item Articulate the collaborative commitments and technological systems that mediate data sharing within archaeological projects
    \item Relate these accepted norms and expectations to the sociotechnical arrangements that frame participation in open data infrastructures
  \end{enumerate}
\end{frame}

\begin{frame}{My Approach}
  \begin{itemize}
    \item I examine archaeological research as social and cultural practice
    \item Especially interested in:
    \begin{itemize}
      \item how objects gain meaning through collective action
      \item the social and technological systems to mediate and encourage cooperative and productive behaviour
      \item tensions that emerge through social and technological re-arrangements
    \end{itemize}
  \end{itemize}
\end{frame}
  

  \begin{frame}{Data Collection and Analysis}
    \begin{columns}[T,onlytextwidth]
      \column{0.45\textwidth}
      \begin{itemize}
        \item Data collected over four years, from 2017 - 2020
        \item Three cases
        \item 90+ hours of work observed
        \item 30+ interviews
        \item On site, museum, dig house, and office settings, both on and off season
        \item Abductive qualitative data analysis methods roughly based on grounded theory
      \end{itemize}

      \column{0.5\textwidth}
    \centering
    \includegraphics[width=4cm,]{images/fig1.png}
  \end{columns}
\end{frame}

\begin{frame}{Archaeological Projects}
  \begin{itemize}
    \item Projects define the scope of legitimate work
    \item Legitimacy afforded based on:
    \begin{enumerate}
      \item warrant afforded by the broader research community
      \item ability to assemble a project system comprising a network of collaborating individuals
    \end{enumerate}
  \end{itemize}

\begin{block}{Basil, Project Director, Case A}
Well, when you make your application for an excavation \dots they gotta make sure it’s up to snuff, quality wise. \textbf{They don’t want problems \dots because it’s not umm professional enough, or it’s asking an inappropriate question.} And so, one of the things we’re required to do is list our personnel. So in listing our personnel \dots \alert{\textbf{you try to dovetail as best as possible, with like, these are our research questions, these are the methods we aim to employ to answer these research questions, and here are the specialists who are capable of wielding those techniques.}}
\end{block}

\end{frame}


\begin{frame}{Divisions of Labour}
  
  \centering

  {
  \setlength{\fboxsep}{0pt}
  \setlength{\fboxrule}{1pt}
  \fbox{\includegraphics[width=5.5cm,]{images/fig8.png}}
  }

  {
  \setlength{\fboxsep}{0pt}
  \setlength{\fboxrule}{1pt}
  \fbox{\includegraphics[width=5.25cm,]{images/fig14.png}}
  }

\end{frame}

\begin{frame}{Terriroriality}
  \begin{block}{Lauren, Trench Supervisor, Case A}
  \dots we are really close with our trenches, especially Theo and I, so we can talk about what’s happening in our trenches, correlate it, and ask each other for opinions. \dots Usually it’s like, umm, Theo sticking his out of his trench and is like, Lauren, do you have a moment? Or me saying, Theo, can you have a look at this? And then umm, we compare, usually we compare, like, our stratigraphy or we look at material, like getting each other’s opinion
  \end{block}
  
  \begin{block}{Basil, Project Director, Case A}
  Jolene’s been here from the get-go [so] she has sort of first dibs on what sort of material she wants to work on
  \end{block}

  \begin{block}{Rufus, Project Director, Case B}
  well the project directors, they devote their lives to these things. And in a way sure, I guess \dots they kinda get first dibs
\end{block}


\end{frame}

\begin{frame}{Agency and Activity Systems: Excavation}

  \begin{block}{Ben, Trench Assistant, Case A}
    Like if I did something that Lauren didn’t like it’s different than when I did something [a different supervisor] didn’t like.
  \end{block}
  
  \begin{block}{Ben, Trench Assistant, Case A}
    I sort of questioned why we have to keep the sections, like in our trench, so like uniform. Like we can’t pick rocks out. Like if it’s sticking out we gotta keep it in there if it’s not-- the section is gonna have a rock sticking into the trench, but they don’t want a rock removed from the trench if it’s gonna create a hole in the section. So it’s just sticking into the trench, and that could be an important lithic, right? So I never understood that, but that’s how it’s done.
  \end{block}

\end{frame}

\begin{frame}{Agency and Activity Systems: Reports}

  \begin{block}{Rufus, Project Director, Case B}
    \dots our department of antiquities publications or whatever, \textbf{as a trench supervisor student, I was never put on reports.} I was never put on those publications. \textbf{Once I became field director, which is obviously under co-director, I was.} And I was given opportunities to take, to be a lead on these things. I understood that from their perspective that \alert{they are bringing over people like trench supervisors, bringing them over, paying their way, room and board, all that stuff, and they had a specific task to do. When they did that task, that information is then absorbed into the project.} If they went above and beyond or wanted to take an active role in something, there was opportunities to do that.
  \end{block}

\end{frame}

\begin{frame}{Agency and Activity Systems: Syntheses}

  \begin{block}{George, Project Director, Case C}
    \dots if you write a paper and you cite somebody and you, you know, you have it in parentheses in the text and in a bibliographic reference at the end, that doesn’t make them a collaborator, just because you cited them \dots But in the case of where you are constantly aggregating more and more and more data, I think it does become difficult, and somewhat, you know, there’s some doubt in my mind that we’re adequate– that were accurately and adequately tracking all of those sources.
  \end{block}

\end{frame}

\begin{frame}{The value of various representations}
  \begin{itemize}
    \item Main point
    \item Examples (show field journals and recording sheets)
  \end{itemize}
\end{frame}

\begin{frame}{Information Systems / Flows}
  Projects are sociotechnical systems, assembled to produce a central record pertaining to a spatially or topically bounded material assemblage
  Projects try to impose a sense of order, to dictate the flow of information
  This is through organizational structures like dictating who responds to who, and through technical protocols

  Different kinds of information are valued in different ways
  The things included in an information system are geared towards producing certain kinds of outputs
  Things excluded from the information system are notably things that do not get you to those desired kinds of outputs
  Are there patterns in who creates or works with these things?
  
\end{frame}

\begin{frame}{Snippets}
  Figures depicting a typical project hierarchy, and how this maps onto the material assemblage; each specialist maintains control over their own material domain
  Highlight the role of reports in transmitting information from specialists to centralizing systems controlled by directors
  Quote from Case 1 about the role and value of trench notebooks
  Notes from Case 1 about the role and value of site tours
  Quote from Case 1 about the need to convince a visiting specialist, and the transformation of this experience into a "personal communication" citation and into word of mouth acceptance of findings
  Quotes from Case 3 on extraneous and potentially embarassing information included in informal records
  

\end{frame}


\begin{frame}{Information Management as Management}
  The systems that manage data do so by managing the labour that produces data

\end{frame}

\begin{frame}
  Cases 1, 2 and 3 on the directors ownership and control over the datasets they govern

\end{frame}

\section{Tensions with Open Data Infrastructures}

\begin{frame}{XXXX}
  Projects remain the fundamental units of concern, those boundaries are maintained
  Power is further concentrated in the directors' position, as the primary contact, and as the recognized owner and controller of the totality of a project's information; the boundaries within a project tend to be rendered invisible
  Datasets are presented as they are in their final state; there is essentially no mention of the mechanisms that governed data flow (cite Isto Huvila and his colleagues); perhaps this is simply a matter of general sense of mutual understanding, a product of institutional isomorphism and immersion within the community of practice (something about seeming completely clueless if you include computational specifications at a certain degree of detail)

\end{frame}

\begin{frame}{Where do we go from here?}
  Goal-oriented data sharing
  Report-oriented publication strategy; assemblage of specialist reports, rather than a single cohesive document
  Provide contact information
  Registry rather than repository mentality
  Accept and acknowledge the complicated imperfections, like fuzzy notions of authorship and spotty inclusion of informal records in digital repositories; imposing structures and standards is not the solution, we should instead treat these as informal aspects of our own cultural practice, something to be talked about freely and reiteratively discussed and reflected upon; there is no need for a formal solution to give artificial and unwanted structure, embrace the anarchy


\end{frame}

\section*{Back matter}

\begin{frame}[standout]
  Questions?
\end{frame}

\begin{frame}
  
\begin{center}You can download this presentation at\\
\url{github.com/zackbatist/DAB23}\end{center}

\begin{center}\doclicenseImage[imagewidth=2cm]\\
\footnotesize \href{http://creativecommons.org/licenses/by-sa/4.0/}{Creative Commons Attribution-ShareAlike 4.0}\end{center}


% \begin{center}\ccbysa\end{center}

\end{frame}

\begin{frame}{References}
  \bibliography{demo}
  \bibliographystyle{abbrv}
\end{frame}

\end{document}