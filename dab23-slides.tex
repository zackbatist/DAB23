\documentclass{beamer}
\setbeamersize{text margin left=5mm,text margin right=5mm} 
\usetheme{metropolis}
\title{Documenting the collaborative commitments that support data sharing within archaeological project collectives}
\date{\today}
\author{Zachary Batist}
\institute{Faculty of Information\\University of Toronto}

\begin{document}
\maketitle

\begin{frame}{Objectives}
  \begin{enumerate}
    \item Articulate various collaborative commitments that frame data sharing among close collaborators on a local level
    \item Relate these norms and expectations to the affordances and limitations that structure participation in open data infrastructures

  \end{enumerate}
  
\end{frame}

\begin{frame}
  \frametitle{Table of Contents}
  \tableofcontents
\end{frame}

\section{My Approach}

\begin{frame}{Archaeology is inherently collaborative}
  \begin{itemize}
    \item Work is distributed across members of project teams
    \item Activities are performed asynchronously and in tandem across time and place
    \item Projects draw from and extend upon prior and concurring work
    \item Professional standards and community expectations dictate how work should be performed and what derived outcomes should look like
  \end{itemize}
\end{frame}

\begin{frame}{Archaeologists produce media to facilitate collaboration}
  \begin{itemize}
    \item Media are externalized representations of ideas or experiences
    \item Include both tangible and computationally useful records, and more malleable means of communication
  \end{itemize}
  
  \metroset{block=fill}
  \begin{columns}[T,onlytextwidth]
    \column{0.45\textwidth}
      \begin{block}{Formal Records}
        \begin{itemize}
          \item Recording sheets
          \item Spreadsheets
          \item Stratigraphic sequences
        \end{itemize}
      \end{block}

    \column{0.4\textwidth}
      \begin{block}{Informal Records}
        \begin{itemize}
          \item Email, phone calls
          \item Field manuals
          \item Stories
        \end{itemize}
      \end{block}
  \end{columns}

  \begin{itemize}
    \item All enable ideas and personal experiences to be shared, accumulated, and rendered comparable
    \item Different kinds of records tend to be used and valued in different ways
  \end{itemize}
  
\end{frame}



\section{Methods and Data}

\begin{frame}{Data Collection}
  \metroset{block=fill}
  \begin{columns}[T,onlytextwidth]
    \column{0.52\textwidth}
    \begin{block}{Observation}
      \begin{itemize}
        \item Video, audio, handwritten notes
        \item 90+ hours of work observed
        \item 150+ hours of recorded footage
      \end{itemize}
    \end{block}

    \begin{block}{Document Analysis}
      \begin{itemize}
        \item Recording sheets
        \item Photos
        \item Labels and tags
        \item Database entries
        \item Spreadsheets
        \item Trench reports ...
      \end{itemize}
    \end{block}

    \column{0.45\textwidth}

    \begin{block}{Retrospective Interviews}
      31 interviews held with 21 distinct participants about:
      \begin{itemize}
        \item the project’s goals
        \item publication strategies
        \item collaboration
        \item teaching and learning ...
      \end{itemize}
    \end{block}

    \begin{block}{Contextual Interviews}
      Explanations and reflections about specific tasks as work proceeded
    \end{block}

  \end{columns}

\end{frame}

% \begin{frame}{Data Collection}
  % \begin{itemize}
    % \item Observational records (video, audio, handwritten notes)
    % \begin{itemize}
      % \item 90 + hours of work observed
      % \item 150 + hours of recorded footage
    % \end{itemize}
    % \item Episodic interviews
    % \begin{itemize}
      % \item Explanations, think-aloud, as work proceeded
    % \end{itemize}
    % \item Retrospective interviews
    % \begin{itemize}
      % \item Thoughts about the project’s goals, publication strategies,
      % \item Collaboration, teaching and learning, etc
      % \item 31 interviews held with 21 distinct participants
    % \end{itemize}
    % \item Document analysis
    % \begin{itemize}
      % \item Forms, photos, labels, DB entries, spreadsheets, trench reports, etc
    % \end{itemize}
  % \end{itemize}
% \end{frame}


\begin{frame}{Data Collection}
  \begin{columns}
    \column{0.55\textwidth}
      \begin{itemize}
        \item Data collected over four years, from 2017 - 2020
        \item Observations conducted on site, in museum, and in dig house settings
        \item Interviews held during summer fieldwork and during the off-season
        \item Abductive qualitative data analysis methods using MaxQDA
        \begin{itemize}
          \item Following Charmaz 2014, \textit{Constructing Grounded Theory}
        \end{itemize}
        \item Names of people and places are pseudonyms
      \end{itemize}

      \column{0.4\textwidth}
      \centering
      \includegraphics[width=2cm,]{Picture2.png}
        \includegraphics[width=5cm]{Picture1.png}

\end{columns}
\end{frame}


\begin{frame}{Cases}
  Case 1
  \begin{itemize}
    \item Longitudinal (2017-2019), carte blanche access to all their records
    \item Prehistoric site in southern Europe, follows fairly traditional methods and data management practices
  \end{itemize}

  Case 2
  \begin{itemize}
    \item Hellenistic site in southern Europe
    \item Very tech-savvy, early adopter of digital tools and methods
    \item Participation spanned one week towards the end of fieldwork in 2019
  \end{itemize}

  Case 3
  \begin{itemize}
    \item A data sharing consortium focused on a specific region
    \item Focused on technical and social challenges re: data sharing
    \item Entirely based on interviews and handwritten notes from 2019-2020
  \end{itemize}
\end{frame}

\section{Boundaries}
\begin{frame}{Archaeological Projects}
  
\end{frame}

\begin{frame}{Divisions of Labour}
  People in different roles hold different amount of creative agency
  They also hold different capability to participate within broader discourse (matroshka doll)
  The boundaries between human and tool agency get very blurred, as people are managed and leveraged to produce targeted outputs
  
\end{frame}

\begin{frame}{Snippets}
  Quote from Case 1 about managing conflict, ensuring no one is stepping on each others' shoes
  Quote from Case 3 about maintaining autonomy
  Quote from Case 3 about the differences between projects
  Quote from Case 2 about the director being the primary actor, and compare with other collaborative systems like film direction and corporate ownership whereby the scope of the company is dictated by what is and is not owned by the owners
  
\end{frame}

\section{Information Management Systems}

\begin{frame}
  Projects are sociotechnical systems, assembled to produce a central record pertaining to a spatially or topically bounded material assemblage
  Projects try to impose a sense of order, to dictate the flow of information
  This is through organizational structures like dictating who responds to who, and through technical protocols

\end{frame}

\begin{frame}{Information Flows}
  Different kinds of information are valued in different ways
  The things included in an information system are geared towards producing certain kinds of outputs
  Things excluded from the information system are notably things that do not get you to those desired kinds of outputs
  Are there patterns in who creates or works with these things?
  
\end{frame}

\begin{frame}{Snippets}
  Figures depicting a typical project hierarchy, and how this maps onto the material assemblage; each specialist maintains control over their own material domain
  Highlight the role of reports in transmitting information from specialists to centralizing systems controlled by directors
  Quote from Case 1 about the role and value of trench notebooks
  Notes from Case 1 about the role and value of site tours
  Quote from Case 1 about the need to convince a visiting specialist, and the transformation of this experience into a "personal communication" citation and into word of mouth acceptance of findings
  Quotes from Case 3 on extraneous and potentially embarassing information included in informal records
  

\end{frame}


\section{Data Governance}

\begin{frame}{Information Management as Management}
  The systems that manage data do so by managing the labour that produces data

\end{frame}

\begin{frame}
  Cases 1, 2 and 3 on the directors ownership and control over the datasets they govern

\end{frame}

\section{Tensions with Open Data Infrastructures}

\begin{frame}{XXXX}
  Projects remain the fundamental units of concern, those boundaries are maintained
  Power is further concentrated in the directors' position, as the primary contact, and as the recognized owner and controller of the totality of a project's information; the boundaries within a project tend to be rendered invisible
  Datasets are presented as they are in their final state; there is essentially no mention of the mechanisms that governed data flow (cite Isto Huvila and his colleagues); perhaps this is simply a matter of general sense of mutual understanding, a product of institutional isomorphism and immersion within the community of practice (something about seeming completely clueless if you include computational specifications at a certain degree of detail)

\end{frame}

\begin{frame}{Where do we go from here?}
  Goal-oriented data sharing
  Report-oriented publication strategy; assemblage of specialist reports, rather than a single cohesive document
  Provide contact information
  Registry rather than repository mentality
  Accept and acknowledge the complicated imperfections, like fuzzy notions of authorship and spotty inclusion of informal records in digital repositories; imposing structures and standards is not the solution, we should instead treat these as informal aspects of our own cultural practice, something to be talked about freely and reiteratively discussed and reflected upon; there is no need for a formal solution to give artificial and unwanted structure, embrace the anarchy


\end{frame}

\section*{Back matter}

\begin{frame}[standout]
  Questions?
\end{frame}

\begin{frame}{XX}
  
Get the source of this theme and the demo presentation from

\begin{center}\url{github.com/matze/mtheme}\end{center}

The theme \emph{itself} is licensed under a \href{http://creativecommons.org/licenses/by-sa/4.0/}{Creative Commons Attribution-ShareAlike 4.0 International License}.

% \begin{center}\ccbysa\end{center}

\end{frame}

\begin{frame}{References}
  \bibliography{demo}
  \bibliographystyle{abbrv}
\end{frame}

\end{document}