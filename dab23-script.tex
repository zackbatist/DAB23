\documentclass{article}
\usepackage[margin=1in]{geometry}
\usepackage{fancyhdr}

\begin{document}
\pagestyle{fancy}
\fancyhead[L]{
  \small Documenting the collaborative commitments that support data sharing within archaeological project collectives\newline
  \small Zack Batist
}
\fancyhead[R]{\vspace{1em}
  \small Digital Archaeology Bern 2023
}
\fancyhead[C]{\vspace{1em}
  \small Presentation Notes
}

\section{Title}
\begin{enumerate}
  \item Hi everyone, glad to be here.
  \item The paper I'm about to present draws from my dissertation, which ins't strictly about open science but is heavily inspired by challenges that digital archaeologists commonly face while engaging with open research data,
  \item namely the awkwardness of opening a spreadsheet and entering a world of tacit assumptions and experiences that differ from your own, and the actions taken to reconcile these disparities.
\end{enumerate}

\section{Objectives}
\begin{enumerate}
  \item I consider uploading and downloading formally arranged information from a digital repository as just one form of data sharing among many other possibilities.
  \item Archaeologists share data all the time, following numerous protocols that don't necessarily involve making the records available 24/7 on the web.
  \item My work basically tries to articulate all the different kinds of data sharing that archaeologists do, while calling attention to the collaborative commitments that they entail and the role of mediating technologies in these interactions.
  % \item For instance, I examine how archaeologists within projects stake ownership over a dataset, which may be expressed in terms of the amount of recognized effort put into the construction of a dataset, associations with a tool used to produce it, one's status within the project hierarchy, or an explicit mandate provided by an authoritative figure or institution.
  \item I think that many of the problems that archaeological information infrastructures have to deal with relate to an inability to grapple with archaeologists' discomfort and apprehension regarding the expansion if their collaborative commitments into new dimensions for which we have not yet established social and professional norms.
  \item So I'm here to shed some light on some potential incongruities between how archaeologists tend to share data on a local and collegial scale, and the very different sociotechnical arrangements that tend to be imposed by open data infrastructures, at least under the paradigm we have opted to pursue.
\end{enumerate}

\section{My Approach}
\begin{enumerate}
  \item First a bit about my approach.
  \item I examine archaeological research as a cultural practice, as work that occurs within a social environment.
  \item I'm less interested in what archaeologists are looking at than how they do it, and crucially, how the meanings they ascertain about things are validated through participation within a collective enterprise.
  \item So I set out to observe and interview archaeologists as they work, and articulate the social and technological circumstances that underlie the formation of a communal data stream.
\end{enumerate}

\section{Data Collection and Analysis}
\begin{enumerate}
  \item I recorded over 90 hours of archaeologists doing their work, interviewed over 21 participants, and examined the documents that archaeologists produce and handle on a regular basis.
  \item I did this over several years, and three cases are represented in my work.
  \item Observations and interviews were held in settings where people were excavating, collecting samples, processing and identifying finds, working at their laptops, giving site tours, and holding casual conversations.
  \item All of the names presented here are pseudonyms, by the way.
  \item I then applied qualitative data analysis methods derived from grounded theory to sort through my vast dataset and articulate some trends and theories that underlie them, some of which are presented here.
\end{enumerate}

\section{Archaeological Projects}
The first thing I want to highlight is the role of professional boundaries in the creation and use of archaeological data.
One kind of boundary that is particular important is the boundary around a project.
A project is granted a license to exclusively examine an archaeological assemblage typically spatial bounded by what we call a site.
Attached to the license is a lot of money, which can be leveraged to produce prestige and advance one's careers and the careers of friends and colleagues
A project director who applied for the license and funding typically has universal control over how that funding is allocated, and is responsible for inviting others to participate in the project.
In effect, he is therefore providing opportunities for people to have access to a scarce resource, namely archaeological finds, and to transform them into legitimate records about the past.
These records, which are manifested as published articles, are the currency through which someone builds their career.
Nice little racket, eh?
Anyway, the control held by the director generally corresponds with the scope of the project as a whole.
Just as with corporate structures, the domain of the firm is bounded by what is owned by the owners.
The database is also controlled by the director, but we will get to that in a moment.

\section{Divisions of Labour (based on what people work with)}
Directors allocate responsibilities over certain material assemblages to specialists and supervisors.
These delegated individuals take their experiences with the materials, transform them into written documents, and pass them along to the director in the form of a report.
They may also share more granular, formally arranged information for tight integration into a centralized database, which can then be retrieved along other relevant data.
In general, however, published articles tend to lack much of this potential for integration, with results being presented on a subdisciplinary basis in turn.
That does not mean that there is no collaboration.
I observed numerous excited conversations between geoarchaeological specialists of various sorts, who plan and strategize where to excavate and collect samples from in a proactive way.
However, this does not translate to a publishable document.
The division of labour that corresponds with material assemblages is only one axis through which labour is divided, and it is in fact an axis that is only useful for allocating who owns what assemblages and data.

\section{Divisions of Labour (based on agency}
This is more about one's ability to participate in a conversation and have the potential to influence the way others work.
In other words, another division of labour is characterized by different kinds of agency.

People in different roles hold different amount of creative agency
They also hold different capability to participate within broader discourse (matroshka doll)
The boundaries between human and tool agency get very blurred, as people are managed and leveraged to produce targeted outputs

\section{Agency and Activity Systems}
Note that all of this is based in context.
A director does not necessarily have creative agency in the experience of a trench. Instead, they hold supportive agency, providing the tools and resources necessary to facilitate good work performed by creative and automated actors. Once the information collected by the trench leaves that setting and enters the dig house or the database, it enters a different functional realm, no longer a product derived from intimate familiarity with a trench, and becoming a generic resource to be leveraged by a publishing member of the team who relies on analyst and computational work environment as supportive agents, who provides him with findings to be put in a narrative presented by the director.
It is noteworthy that only the activity contexts where upper management wields creative agency hold the potential to garner some legitimately valuable outcome. Excavators produce records that are not useful in themselves. Even field journals, which are actually much richer and informative than other media produced elsewhere in any given project, are left out of the centralized database and generally ignored and not taken seriously (unless they were written by some long dead white man from the colonial era, such as Flinders Petrie and Arthur Evans).

\section{Information Flows}
Projects are sociotechnical systems, assembled to produce a central record pertaining to a spatially or topically bounded material assemblage
Projects try to impose a sense of order, to dictate the flow of information
This is through organizational structures like dictating who responds to who, and through technical protocols

Different kinds of information are valued in different ways
The things included in an information system are geared towards producing certain kinds of outputs
Things excluded from the information system are notably things that do not get you to those desired kinds of outputs
Are there patterns in who creates or works with these things?

\section{Information Management as Management}
The systems that manage data do so by managing the labour that produces data

\section{Tensions with Open Data Infrastrcutures}
Projects remain the fundamental units of concern, those boundaries are maintained
Power is further concentrated in the directors' position, as the primary contact, and as the recognized owner and controller of the totality of a project's information; the boundaries within a project tend to be rendered invisible
Datasets are presented as they are in their final state; there is essentially no mention of the mechanisms that governed data flow (cite Isto Huvila and his colleagues); perhaps this is simply a matter of general sense of mutual understanding, a product of institutional isomorphism and immersion within the community of practice (something about seeming completely clueless if you include computational specifications at a certain degree of detail)

\section{Where do we go from here?}
Goal-oriented data sharing
Report-oriented publication strategy; assemblage of specialist reports, rather than a single cohesive document
Provide contact information
Registry rather than repository mentality
Accept and acknowledge the complicated imperfections, like fuzzy notions of authorship and spotty inclusion of informal records in digital repositories; imposing structures and standards is not the solution, we should instead treat these as informal aspects of our own cultural practice, something to be talked about freely and reiteratively discussed and reflected upon; there is no need for a formal solution to give artificial and unwanted structure, embrace the anarchy

\section{Questions?}

\end{document}