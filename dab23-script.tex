\documentclass{article}
\usepackage[margin=1in]{geometry}
\usepackage{fancyhdr}
\usepackage{soul}

\begin{document}
\pagestyle{fancy}
\fancyhead[L]{
  \small Documenting the collaborative commitments that support data sharing within archaeological project collectives\newline
  \small Zack Batist
}
\fancyhead[R]{\vspace{1em}
  \small Digital Archaeology Bern 2023
}
\fancyhead[C]{\vspace{1em}
  \small Presentation Notes
}

\section{Title}
\begin{enumerate}
  \item Hi everyone, glad to be here.
  \item The paper I'm about to present draws from my dissertation, which ins't strictly about open science but is heavily inspired by challenges that digital archaeologists commonly face while engaging with open research data,
  \item namely the awkwardness of opening a spreadsheet and entering a world of tacit assumptions and experiences that differ from your own, and the actions taken to reconcile these disparities.
\end{enumerate}

\section{Objectives}
\begin{enumerate}
  \item I consider uploading and downloading formally arranged information from a digital repository as just one form of data sharing among many other possibilities.
  \item Archaeologists share data all the time, following numerous protocols that don't necessarily involve making the records available 24/7 on the web.
  \item My work basically tries to articulate all the different kinds of data sharing that archaeologists do, while calling attention to the collaborative commitments that they entail and the role of mediating technologies in these interactions.
  % \item For instance, I examine how archaeologists within projects stake ownership over a dataset, which may be expressed in terms of the amount of recognized effort put into the construction of a dataset, associations with a tool used to produce it, one's status within the project hierarchy, or an explicit mandate provided by an authoritative figure or institution.
  \item I think that many of the problems that archaeological information infrastructures have to deal with relate to an inability to grapple with archaeologists' discomfort and apprehension regarding the expansion if their collaborative commitments into new dimensions for which we have not yet established social and professional norms.
  \item So I'm here to shed some light on some potential incongruities between how archaeologists tend to share data on a local and collegial scale, and the very different sociotechnical arrangements that tend to be imposed by open data infrastructures, at least under the paradigm we have opted to pursue.
\end{enumerate}

\section{My Approach}
\begin{enumerate}
  \item First a bit about my approach.
  \item I examine archaeological research as a cultural practice, as work that occurs within a social environment.
  \item I'm less interested in what archaeologists are looking at than how they do it, and crucially, how the meanings they ascertain about things are validated through participation within a collective enterprise.
  \item So I set out to observe and interview archaeologists as they work, and articulate the social and technological circumstances that underlie the formation of a communal data stream.
\end{enumerate}

\section{Data Collection and Analysis}
\begin{enumerate}
  \item I recorded over 90 hours of archaeologists doing their work, conducted over 30 interviews with more than 20 participants, and examined the documents that archaeologists produce and handle on a regular basis.
  \item I reviewed over 150 hours of recorded footage, based on observation alone, from my three cameras, which I placed in different parts of a trench, and sometimes even on participants' foreheads.
  \item I did this over several years, and three cases are represented in my work.
  \item Observations and interviews were held in settings where people were excavating, collecting samples, processing and identifying finds, working at their laptops, giving site tours, and holding casual conversations.
  \item However, most of the evidence I cite here today is derived from retrospective interviews, since this is the most direct way in which attitudes on collaboration and data sharing are expressed.
  % \item I then applied qualitative data analysis methods derived from grounded theory to sort through my vast dataset and articulate some trends and theories that underlie them, some of which are presented here.
  \item All of the names presented here are pseudonyms, by the way.
\end{enumerate}

\section{Archaeological Projects}
\begin{enumerate}
  \item The first thing I want to highlight is the role of professional boundaries in the creation and use of archaeological data.
  \item One kind of boundary that is particular important is the boundary around a project.
  \item A project is granted license to examine an archaeological assemblage, which is spatial bounded by what we call a site.
  \item The process of obtaining a license requires that applicants engage with research communities to justify their research objectives, and that projects clearly articulate a plan of action for administrative entities.
  \item For instance, as Basil, the project director for my first case noted, he had to frame his research questions in a strategic way.
  \item Moreover, he had to show that he was capable of addressing the research questions, which primarily involves tapping into his network of collaborators.
  \item It is thus necessary to engage in two discursive fronts, one external to the project, and the other internal.
  \item With regards to the latter, work is delegated to people holding various roles, who operate according to professional expectations regarding what their role entails.

\end{enumerate}

\section{Divisions of Labour}
\begin{enumerate}
  \item In the interest of time, I won't elaborate much on the divisions of specific tasks, i.e. the makeup of project hierarchies, which I write about in much greater depth in my dissertation.
  \item Instead, I'll focus on a few significant factors relating to the distribution of agency in the production and distribution of tangible research outputs, such as internal records and published documents and datasets.
\end{enumerate}
  
\section{Territoriality}
\begin{enumerate}
  \item First, the allocation of work is perceived in a territorial way.
  \item This occurs all across a project.
  \item For example, Lauren, a trench supervisor at Case A, discussed her collegial relationship with Theo, work supervises work on another nearby trench.
  \item They regularly share insights on each other's work, but it remains clear that each person retains control over their respective trenches.
  \item With regards to specialist work, Basil, the project director, feels a need to ensure that work is distributed fairly and with clear intent.
  \item The allocation of research materials, which can be leveraged to produce published research outputs, has tangible effects on people's careers, and is therefore taken very seriously.
  \item With regards to projects as a whole, Rufus, who co-directs Case B, identifies project directors as all-encompassing owners, due to the effort they expend getting things off the ground.
  \item Of course, their territory only pertains to projects that they lead themselves, which reifies the boundaries between projects.
  \item Like with corporate structures, a firm's boundaries are defined by what is owned by the owners.
  \item The main takeaway from this is that, based on my observations, the archaeological project commons comprises a series of contributions made by individuals, who hold their own stakes and expectations.

\end{enumerate}

\section{Agency and Activity Systems}
\begin{enumerate}
  \item Another factor is that people in differing positions exhibit different kinds of agency, depending on their roles within various activities.
  \item For instance, trench supervisors wield creative agency in their trenches, and leverage their assistants as automatons to support their creation of context recording sheets and trench reports.
  \item Assistants may be left in the dark about why they do things in certain ways, though when they acquire this knowledge they may eventually take on creative agency over their own trenches.
  \item This was the case with Ben, who started off as a trench assistant, but who became a supervisor when he returned the next season.
  \item He lamented that he had to work his assistants as mere tools, and recognized this imperative as being sourced to overall project structures.
  \item In turn, he has commitments to the project director, who draws from the reports that Ben produces to assemble overarching articles or reports pertaining to the project as a whole.
  \item For the director, each supervisor's contribution is genericized in the same was as the assistants are genericized for the supervisors.
  \item They are rendered as tools directed towards certain ends that they themselves control.
  \item We may extend this even further by identifying people who integrate data derived from multiple projects as creative agents who leverage the work produced by generic project directors in a similar way.
  \item To deal with the insecurity that project directors sense when sharing the fruits of their labour, George set up Case C, which is a data sharing consortium based on a specific region, on top of existing personal and professional collaborations.
  \item Everyone already knows everyone, which makes it very difficult to genericize anyone's contributions or co-opt others' work without their direct participation.
  
  \item It is notable that in each of these levels, the expectation of authorship, or general recognition of credit, gradually disappears.
  \item Internal trench reports, published project reports, and synthetic studies each exclude the people they depend upon, unless this effect is actively mitigated, as in the last case I mentioned.
  \item In any given activity, a certain set of individuals hold creative agency, or the power to assemble people and tools in a manner that suits their goals.
\end{enumerate}

\section{The Value of Various Representations}
\begin{enumerate}
  \item It is also noteworthy that only activity where upper management wields creative agency hold the potential to garner some legitimately valuable outcome.
  \item For instance, excavators produce records that are not useful in themselves, and the products of their work must be integrated and analyzed by publishing members of the team in order to yield valued findings.
  \item Once the information collected in the trench leaves that setting and enters the dig house or the database, it enters a different functional realm, no longer a product derived from intimate familiarity with a trench, and becomes a generic resource to be leveraged by a publishing member of the team.
  \item Excavators, on the other hand, produce records that are not useful in themselves.
  \item For instance, field journals, which are actually much richer and informative than other media produced elsewhere in any given project, are left out of the centralized database, and are treated as dead-end representations of archaeological materials, along with sketches, stories, and site tours.
  \item examples: Liz and Theo on the use of notebooks
  \item example: something about the lawsuit mentioned by George and Barry, whereby work that was created in anticipating a dead-end was then extended into more public-facing positions

\end{enumerate}

\section{Information Systems / Flows}
\begin{enumerate}
  \item The boundaries between human and tool agency gets really blurred when dealing with a broad system comprising many kinds of activities that occur sequentually and in tandem

  \item Information systems play a significant role in bringing some semblance of order to such messy collective efforts.
  \item Supportive agents are those who enable work to proceed without necessarily being leveraged in the work activities themselves.

  \item Projects are sociotechnical systems, assembled to produce a central record pertaining to a spatially or topically bounded material assemblage.
  \item Projects try to impose a sense of order, to dictate the flow of information.
  \item They do this by situating data and labour as part of information systems that dictate how information should be structured, which afford data with certain functional properties.
  \item Different kinds of information are valued in different ways, as the information system is tailored to process data in ways that will eventually generate certain kinds of outputs, namely publishable articles.
  \item Workflows
  \item Things excluded from the information system are notably things that do not get you to those desired kinds of outputs, i.e. things like field journals, sketches, draft records, and personal or informal recollections.
  \item These things, despite not making it into the publishable record, are crucial for making sense of formal records, by providing situational context to the things that were formally recorded.
\end{enumerate}

\section{Information Management as Management}
\begin{enumerate}
  \item This contributes to my second argument, that it is impossible to distinguish between the management of data and the management of labour implelented to produce them.
  \item Information systems mask professional expectations and collaborative commitments within and between project systems.
\end{enumerate}

\section{Data governance}
\begin{enumerate}
  \item Who holds ultimate control?
  \item What is the role of digital information systems in enforcing regimes of ownership and control?
  \item Is it possible to have a more egalitarian system in knowledge production?
  \item Can this be done via technological systems alone, or does this require a broader sociopolitical transformation as well?
  \item That's why they call this sociotechnical, the fact that these weave into each other in co-dependent ways.
\end{enumerate}

\section{Tensions with Open Data Infrastrcutures}
\begin{enumerate}
  \item Projects remain the fundamental units of concern, those boundaries are maintained
  \item Power is further concentrated in the directors' position, as the primary contact, and as the recognized owner and controller of the totality of a project's information; the boundaries within a project tend to be rendered invisible
  \item Datasets are presented as they are in their final state; there is essentially no mention of the mechanisms that governed data flow (cite Isto Huvila and his colleagues); perhaps this is simply a matter of general sense of mutual understanding, a product of institutional isomorphism and immersion within the community of practice (something about seeming completely clueless if you include computational specifications at a certain degree of detail)
\end{enumerate}

\section{Where do we go from here?}
\begin{enumerate}
  \item Goal-oriented data sharing
  \item Report-oriented publication strategy; assemblage of specialist reports, rather than a single cohesive document
  \item Provide contact information
  \item Registry rather than repository mentality
  \item Accept and acknowledge the complicated imperfections, like fuzzy notions of authorship and spotty inclusion of informal records in digital repositories; imposing structures and standards is not the solution, we should instead treat these as informal aspects of our own cultural practice, something to be talked about freely and reiteratively discussed and reflected upon; there is no need for a formal solution to give artificial and unwanted structure, embrace the anarchy
\item \end{enumerate}

\section{Questions?}

\end{document}