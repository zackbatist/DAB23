\documentclass{article}
\usepackage[margin=1in]{geometry}
\usepackage{fancyhdr}
\usepackage{soul}

\begin{document}
\pagestyle{fancy}
\fancyhead[L]{
  \small Documenting the collaborative commitments that support data sharing within archaeological project collectives\newline
  \small Zack Batist
}
\fancyhead[R]{\vspace{1em}
  \small Digital Archaeology Bern 2023
}
\fancyhead[C]{\vspace{1em}
  \small Presentation Notes
}

\section{Title}
\begin{enumerate}
  \item Hi everyone, glad to be here.
  \item The paper I'm about to present draws from my dissertation, which ins't strictly about open science but is heavily inspired by challenges that digital archaeologists commonly face while engaging with open research data,
  \item namely the awkwardness of opening a spreadsheet and entering a world of tacit assumptions and experiences that differ from your own, and the actions taken to reconcile these disparities.
\end{enumerate}

\section{Objectives}
\begin{enumerate}
  \item I consider uploading and downloading formally arranged information to and from a digital repository as just one kind of data sharing, which exists among a range of other additional collaborative actions.
  \item Archaeologists share data all the time through alternative means, following social and technological protocols that don't necessarily involve making the records available 24/7 on the web.
  \item My work basically tries to articulate these ways in which archaeologists already share data, which do not necessarily rely on the global information infrastructures that "open archaeology" is concerned with developing.
  \item More specifically, I call attention to the collaborative commitments that acts of sharing entail, and the roles of mediating technologies in these interactions.
  \item I think that many of the problems that we face when developing large-scale, web-based, open data infrastructures relate to our inability to grapple with archaeologists' discomfort and apprehension regarding the expansion if their collaborative commitments into new dimensions, for which the discipline has not yet established social and professional norms.
  \item So I'm here to shed some light on some incongruities between how archaeologists tend to share data on a local and collegial scale, and the very different sociotechnical arrangements that are imposed by open data infrastructures, at least under the paradigm we have opted to pursue.
\end{enumerate}

\section{My Approach}
\begin{enumerate}
  \item First a bit about my approach.
  \item I examine archaeological research as a cultural practice, as work that occurs within a social environment.
  \item I'm less interested in what archaeologists are looking at than how they do it, and crucially, how the meanings they ascertain about things are validated through participation within a collective enterprise.
  \item I'm specifically interested in examining how meanings are derived through collective action, how social and technolgical systems mediate and encourage behaviour targeted towards collectively-held goals, and the tensions that emerge through these cooperative systems.
  \item So I set out to observe and interview archaeologists as they work, and articulate the social and technological circumstances that underlie the formation of a communal data stream.
\end{enumerate}

\section{Data Collection and Analysis}
\begin{enumerate}
  \item I recorded over 150 hours of audio and video footage, based over 90 hours of observation, using three cameras that I placed in different parts of archaeological work environments, including on participants' foreheads.
  \item I conducted over 30 interviews with more than 20 participants.
  \item And examined the documents that archaeologists produce and handle on a regular basis.
  \item I did this over several years, and three cases are represented in my work.
  \item Observations and interviews were held in settings where people were excavating, collecting samples, processing and identifying finds, working at their laptops, giving site tours, and holding casual conversations.
  \item However, most of the evidence I cite here today is derived from retrospective interviews, since this is the most direct way in which archaeologists expressed their attitudes on data sharing and collaboration.
  \item For the sake of keeping good time, I will not go into detail on my data collection and analysis methods, but I'm happy to discuss this in greater depth later on or during the break.
  \item All of the names presented here are pseudonyms, by the way.
\end{enumerate}

\section{Archaeological Projects}
\begin{enumerate}
  \item The first thing I want to highlight is the role of professional boundaries in the creation and use of archaeological data.
  \item One important boundary is that which distinguishes an archaeological project.
  \item A project is granted license to examine an archaeological assemblage, which is spatially bounded by what we tend to call a site.
  \item In order to obtain a license, applicants are required to engage with research communities to justify their research objectives, and submit their plans of action for review by administrative entities.
  \item For instance, Basil, the project director for my first case, noted that he had to frame his research questions in a strategic way.
  \item He had to ensure that his questions were appropriate and feasible to achieve,
  \item which primarily involves tapping into his network of collaborators.
  \item It is thus necessary to engage in two discursive fronts:
  \item one external to the project, dealing with the community at large,
  \item and the other being internal, dealing with participating collaborators who will help derive archaeological meaning from the recovered materials.
  \item With regards to the latter, work is delegated to people holding various roles, who operate according to professional expectations that inform what their role entails.

\end{enumerate}

\section{Divisions of Labour}
\begin{enumerate}
  \item In the interest of time, I won't elaborate much on the divisions of specific tasks,
  \item i.e. the makeup of project hierarchies, which I write about in much greater depth in my dissertation.
  \item Instead, I'll focus on a few significant factors relating to the distribution of agency in the production and distribution of tangible research outputs, such as internal records and published documents and datasets.
\end{enumerate}
  
\section{Territoriality}
\subsection{Territoriality 1}
\begin{enumerate}
  \item First, the allocation of work is perceived in a territorial way.
  \item This occurs all across a project.
  \item For example, Lauren, a trench supervisor at Case A, discussed her collegial relationship with Theo, work supervises work on another nearby trench.
  \item They regularly share insights on each other's work, but it's clear that each person retains control over their respective trenches.
  
\end{enumerate}

\subsection{Territoriality 2}
\begin{enumerate}
  \item With regards to specialist work, Basil, the project director, feels a need to ensure that work is distributed fairly and with clear intent.
  \item He sees that specialists, especially those early in their careers, are anxious about retaining control over a particular slice of research materials, and he tries to be very upfront about his expectations regarding what each person is responsible for.
  \item The allocation of research materials, which can be leveraged to produce published research outputs, has tangible effects on people's careers, and Basil therefore takes this very seriously.
  \item At the same time, he rewards seniority and uses the promise of access to research materials as a way of enticing more involved participation in the project.
  
\end{enumerate}
  
\subsection{Territoriality 3}
\begin{enumerate}
  \item With regards to projects as a whole, Rufus, who co-directs Case B, identifies project directors as the most invested stakeholders, due to the effort they expend getting things off the ground.
  \item Of course, their territory only pertains to projects that they lead themselves, which reifies the boundaries between projects.
  \item Like with corporate firms, a project's boundaries are defined by what is owned by the owners slash directors.
  
\end{enumerate}
  
\subsection{Summary: Territoriality}
\begin{enumerate}
  \item The main takeaway is that, based on my observations, archaeological projects are collective endeavours comprised of a series of contributions made by individuals, who hold their own stakes, priorities, and expectations.

\end{enumerate}

\section{Agency and Activity Systems}
\subsection{Agency and Activity Systems 1}
\begin{enumerate}
  \item Another factor is that people in differing positions exhibit different kinds of agency, depending on their roles within various activities.
  \item For instance, trench supervisors wield creative agency in their trenches, and leverage their assistants as automatons to support their creation of context recording sheets and trench reports.
  \item Assistants may be left in the dark about why they do things in certain ways, though when they acquire this knowledge they may eventually take on creative agency over their own trenches.
  \item This was the case with Ben, who started off as a trench assistant.
  \item He simply did what he was told, and effectively worked as his supervisors' digging instrument.
  \item When he came back the next season as a trench supervisor, he lamented that he had to work his assistants in a similar way, and recognized that this imperative is derived from his role as a quick and efficient collector of archaeological materials.
\end{enumerate}

\subsection{Agency and Activity Systems 2}
\begin{enumerate}
  \item Ben then commits his labour to the project director.
  \item The director draws from the reports that Ben and other supervisors produce, and assembles an overarching article or monograph pertaining to the project as a whole.
  \item For the director, each supervisor's contribution is genericized in the same was as the assistants are genericized for the supervisors.
  \item Supervisors are thus rendered as tools that the director mobilizes in order to complete activities in which he is the creative agent.
\end{enumerate}

\subsection{Agency and Activity Systems 3}
\begin{enumerate}
  \item We may extend this even further by identifying people who integrate data derived from multiple projects as creative agents in their own activity systems, and which leverage the work that project directors produce in a similar way.
  \item To deal with the insecurity that project directors sense when sharing the fruits of their labour, George established a data sharing consortium based on a specific region, atop existing personal and professional collaborations.
  \item In this consortium, Everyone already knows everyone, which makes it very difficult to genericize anyone's contributions or co-opt others' work without their direct participation.
\end{enumerate}

\subsection{Summary: Agency and Activity Systems}
\begin{enumerate}
  \item It is notable that in each of these levels, the expectation of authorship, or general recognition of credit, gradually disappears.
  \item Internal trench reports, published project reports, and synthetic studies, each exclude the people they depend upon, unless this effect is actively mitigated, as in the last case I mentioned.
  \item In any given activity, a certain set of individuals hold creative agency, or the power to assemble people and tools in a manner that suits their goals.
\end{enumerate}

\section{Formal Records and Situated Representations}
\begin{enumerate}
  \item Archaeological projects try to bring order to distributed labour through use of data management systems.
  \item These systems are designed to integrate data collected in various settings and make them amenable for retrieval and analysis.
  \item However, as tools designed to produce certain targeted outcomes, i.e. published articles that rely on formal analysis based on discrete and formally arranged data, these data management systems select for and arrange information that meets these goals.
  \item They therefore include only a small range of information generated throughout a project, things that can be digested by a database and subsequently analysed in a way that seems amenable for publication.
  \item This means that data management systems tend to exclude information that does not conform to such rigid standards.
  \item Things like journal entries, sketches, incomplete drafts, or personal recollections, which I like to refer to as "situated representations", are relatively intangible and unstable, and do not directly contribute to the outcomes that information systems are designed to facilitate.
\end{enumerate}

\subsection{Data Management Systems}
\begin{enumerate}
  \item It is also noteworthy that only activities where upper management wields creative agency hold the potential to garner some legitimately valuable outcome.
  \item For instance, excavators produce records that are not useful in themselves, and the products of their work must be integrated and analyzed by publishing members of the team in order to yield valued findings.
  \item Once the information collected in the trench leaves that setting and enters the dig house or the database, it enters a different functional realm, no longer a product derived from intimate familiarity with a trench, and becomes a generic resource to be leveraged by a publishing member of the team.
\end{enumerate}

\subsection{Data Management is Management}
\begin{enumerate}
  \item It is therefore impossible to distinguish between the management of data and the management of labour that produced them.
  \item Information systems, including digital and analog components, mask the professional expectations and collaborative commitments that encourage productive collective efforts.
\end{enumerate}

\section{exceptions}
\begin{enumerate}
  \item However, I identify two kinds of records, field journals and site tours, that manifest tensions between the prioritization of formal records and the lack of technical and infrastructural support for communicating situated experiences.
  \item 
\end{enumerate}




\section{Data governance}
\begin{enumerate}
  \item Who holds ultimate control?
  \item What is the role of digital information systems in enforcing regimes of ownership and control?
  \item Is it possible to have a more egalitarian system in knowledge production?
  \item Can this be done via technological systems alone, or does this require a broader sociopolitical transformation as well?
  \item That's why they call this sociotechnical, the fact that these weave into each other in co-dependent ways.
\end{enumerate}

\section{Tensions with Open Data Infrastrcutures}
\begin{enumerate}
  \item Projects remain the fundamental units of concern, those boundaries are maintained
  \item Power is further concentrated in the directors' position, as the primary contact, and as the recognized owner and controller of the totality of a project's information; the boundaries within a project tend to be rendered invisible
  \item Datasets are presented as they are in their final state; there is essentially no mention of the mechanisms that governed data flow (cite Isto Huvila and his colleagues); perhaps this is simply a matter of general sense of mutual understanding, a product of institutional isomorphism and immersion within the community of practice (something about seeming completely clueless if you include computational specifications at a certain degree of detail)
\end{enumerate}

\section{Where do we go from here?}
\begin{enumerate}
  \item Goal-oriented data sharing
  \item Report-oriented publication strategy; assemblage of specialist reports, rather than a single cohesive document
  \item Provide contact information
  \item Registry rather than repository mentality
  \item Accept and acknowledge the complicated imperfections, like fuzzy notions of authorship and spotty inclusion of informal records in digital repositories; imposing structures and standards is not the solution, we should instead treat these as informal aspects of our own cultural practice, something to be talked about freely and reiteratively discussed and reflected upon; there is no need for a formal solution to give artificial and unwanted structure, embrace the anarchy
\item \end{enumerate}

\section{Questions?}




\end{document}'
