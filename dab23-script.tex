\documentclass{article}
\usepackage[margin=1in]{geometry}
\usepackage{fancyhdr}
\usepackage{soul}

\begin{document}
\pagestyle{fancy}
\fancyhead[L]{
  \small Documenting the collaborative commitments that support data sharing within archaeological project collectives\newline
  \small Zack Batist
}
\fancyhead[R]{\vspace{1em}
  \small Digital Archaeology Bern 2023
}
\fancyhead[C]{\vspace{1em}
  \small Presentation Notes
}

\section{Title}
\begin{enumerate}
  \item Hi everyone, glad to be here.
  \item The paper I'm about to present draws from my dissertation, which ins't strictly about open science but is heavily inspired by challenges that digital archaeologists commonly face while engaging with open research data,
  \item namely the awkwardness of opening a spreadsheet and entering a world of tacit assumptions and experiences that differ from your own, and the actions taken to reconcile these disparities.
\end{enumerate}

\section{Objectives}
\begin{enumerate}
  \item I consider uploading and downloading formally arranged information to and from a digital repository as just one kind of data sharing, which exists among a range of other additional collaborative actions.
  \item Archaeologists share data all the time through alternative means, following social and technological protocols that don't necessarily involve making the records available 24/7 on the web.
  \item My work basically tries to articulate these ways in which archaeologists already share data, which do not necessarily rely on the global information infrastructures that ``open archaeology'' is concerned with developing.
  \item More specifically, I call attention to the collaborative commitments that acts of sharing entail, and the roles of mediating technologies in these interactions.
  \item I think that many of the problems that we face when developing large-scale, web-based, open data infrastructures relate to our inability to grapple with archaeologists' discomfort and apprehension regarding the expansion if their collaborative commitments into new dimensions, for which the discipline has not yet established social and professional norms.
  \item So I'm here to shed some light on some incongruities between how archaeologists tend to share data on a local and collegial scale, and the very different sociotechnical arrangements that are imposed by open data infrastructures, at least under the paradigm we have opted to pursue.
\end{enumerate}

\section{My Approach}
\begin{enumerate}
  \item First a bit about my approach.
  \item I examine archaeological research as a cultural practice, as work that occurs within a social environment.
  \item I'm less interested in what archaeologists are looking at than how they do it, and crucially, how the meanings they ascertain about things are validated through participation within a collective enterprise.
  \item I'm specifically interested in examining how meanings are derived through collective action, how social and technolgical systems mediate and encourage behaviour targeted towards collectively-held goals, and the tensions that emerge through these cooperative systems.
  \item So I set out to observe and interview archaeologists as they work, and articulate the social and technological circumstances that underlie the formation of a communal data stream.
\end{enumerate}

\section{Data Collection and Analysis}
\begin{enumerate}
  \item I recorded over 150 hours of audio and video footage, based over 90 hours of observation, using three cameras that I placed in different parts of archaeological work environments, including on participants' foreheads.
  \item I conducted over 30 interviews with more than 20 participants.
  \item And examined the documents that archaeologists produce and handle on a regular basis.
  \item I did this over several years, and three cases are represented in my work.
  \item Observations and interviews were held in settings where people were excavating, collecting samples, processing and identifying finds, working at their laptops, giving site tours, and holding casual conversations.
  \item However, most of the evidence I cite here today is derived from retrospective interviews, since this is the most direct way in which archaeologists expressed their attitudes on data sharing and collaboration.
  \item For the sake of keeping good time, I will not go into detail on my data collection and analysis methods, but I'm happy to discuss this in greater depth later on or during the break.
  \item All of the names presented here are pseudonyms, by the way.
\end{enumerate}

\section{Archaeological Projects}
\begin{enumerate}
  \item The first thing I want to highlight is the role of professional boundaries in the creation and use of archaeological data.
  \item One important boundary is that which distinguishes an archaeological project.
  \item A project is granted license to examine an archaeological assemblage, which is spatially bounded by what we tend to call a site.
  \item In order to obtain a license, applicants are required to engage with research communities to justify their research objectives, and submit their plans of action for review by administrative entities.
  \item For instance, Basil, the project director for my first case, noted that he had to frame his research questions in a strategic way.
  \item He had to ensure that his questions were appropriate and feasible to achieve,
  \item which primarily involves tapping into his network of collaborators.
  \item It is thus necessary to engage in two discursive fronts:
  \item one external to the project, dealing with the community at large,
  \item and the other being internal, dealing with participating collaborators who will help derive archaeological meaning from the recovered materials.
  \item With regards to the latter, work is delegated to people holding various roles, who operate according to professional expectations that inform what their role entails.

\end{enumerate}

\section{Divisions of Labour}
\begin{enumerate}
  \item In the interest of time, I won't elaborate much on the divisions of specific tasks,
  \item i.e. the makeup of project hierarchies, which I write about in much greater depth in my dissertation.
  \item Instead, I'll focus on a few significant factors relating to the distribution of agency in the production and distribution of tangible research outputs, such as internal records and published documents and datasets.
\end{enumerate}
  
\section{Territoriality}
\subsection{Territoriality 1}
\begin{enumerate}
  \item First, the allocation of work is perceived in a territorial way.
  \item This occurs all across a project.
  \item For example, Lauren, a trench supervisor at Case A, discussed her collegial relationship with Theo, work supervises work on another nearby trench.
  \item They regularly share insights on each other's work, but it's clear that each person retains control over their respective trenches.
  
\end{enumerate}

\subsection{Territoriality 2}
\begin{enumerate}
  \item With regards to specialist work, Basil, the project director, feels a need to ensure that work is distributed fairly and with clear intent.
  \item He sees that specialists, especially those early in their careers, are anxious about retaining control over a particular slice of research materials, and he tries to be very upfront about his expectations regarding what each person is responsible for.
  \item The allocation of research materials, which can be leveraged to produce published research outputs, has tangible effects on people's careers, and Basil therefore takes this very seriously.
  \item At the same time, he rewards seniority and uses the promise of access to research materials as a way of enticing more involved participation in the project.
  
\end{enumerate}
  
\subsection{Territoriality 3}
\begin{enumerate}
  \item With regards to projects as a whole, Rufus, who co-directs Case B, identifies project directors as the most invested stakeholders, due to the effort they expend getting things off the ground.
  \item Of course, their territory only pertains to projects that they lead themselves, which reifies the boundaries between projects.
  \item Like with corporate firms, a project's boundaries are defined by what is owned by the owners slash directors.
\end{enumerate}
  
\subsection{Summary: Territoriality}
\begin{enumerate}
  \item The main takeaways are that, based on my observations, archaeological projects are collective endeavours comprised of a series of contributions made by individuals, who hold their own stakes, priorities, and expectations.
  \item Projects take care to ensure that work is fulfilling for both the individual and the collective.
  \item I identify the norms that archaeologists settle upon, which govern acceptable interactive behaviour among people working as a team, as collaborative commitments, which are more clearly manifested when we consider the roles that people play when advancing specific activities.
\end{enumerate}

\section{Agency and Activity Systems}
\subsection{Agency and Activity Systems 1}
\begin{enumerate}
  \item For instance, trench supervisors wield creative agency in their trenches, and leverage their assistants as tools to support the creation of context recording sheets and trench reports.
  \item Assistants such as Ben, as in the example on screen, may be left in the dark about why they do things in certain ways.
  \item However, when they acquire this understanding they may eventually take on creative agency over their own trenches.
  \item Ben started off as a trench assistant, and in that role he simply did what he was told to do.
  \item He effectively worked as his supervisors' digging instrument.
  \item But when he came back the next season, he became a trench supervisor, and this new role gave him a new imperative to fill out recording sheets.
  \item He lamented that he had to work his assistants in a similar way as he was worked the previous season.
  \item He recognized that his position is that of a quick and efficient collector of archaeological materials, and a producer of preliminary reports, which he then commits to the finds specialists and project director for further study and synthesis.
\end{enumerate}

\subsection{Agency and Activity Systems 2}
\begin{enumerate}
  \item Project directors draw from these trench reports and assembles overarching articles or monographs pertaining to the project as a whole.
  \item For the director, each supervisor's contribution is genericized in the same was as the assistants are genericized for the supervisors.
  \item Supervisors are rendered as tools that the director mobilizes in order to complete activities in which he is the creative agent.
\end{enumerate}

\subsection{Agency and Activity Systems 3}
\begin{enumerate}
  \item We may extend this even further by identifying people who integrate data derived from multiple projects as creative agents in their own activity systems, and which leverage the work that project directors produce in a similar way.
  \item To deal with the insecurity that project directors sense when sharing the fruits of their labour, George established a data sharing consortium built atop existing personal and professional collaborations among archaeologists working in a specific region.
  \item In this consortium, everyone already knows everyone, which makes it very difficult to genericize anyone's contributions or co-opt others' work without their direct participation.
  \item This consortium therefore maintains the collaborative commitments that back each project's autonomy.
\end{enumerate}

\subsection{Summary: Agency and Activity Systems}
\begin{enumerate}
  \item It is notable that in each of these levels, the expectation of authorship, or general recognition of credit, gradually disappears.
  \item Internal trench reports, published project reports, and synthetic studies, each exclude the people they depend upon, unless this effect is actively mitigated, as in the last case I mentioned.
  \item In any given activity, a certain set of individuals hold creative agency, or the power to assemble people and tools in a manner that suits their goals.
\end{enumerate}

\section{Formal Records and Situated Representations}
\begin{enumerate}
  \item Archaeological projects also try to facilitate productive collaborative behaviour by implementing information systems comprising both digital and analog components, which bring together records produced across the project team.
  \item These are directed towards specific ends, namely the integration of information collected across various research contexts.
  \item For example, the observations recorded on a context recording sheet during excavation are linked up with analysis of finds or samples retrieved in that context. \textbf{[cite greg]}
  \item Certain kinds of integrations are afforded based on the kinds of records that these systems are able to parse.
  \item In my observations, only information formally arranged on recording sheets or in spreadsheets are transferred to a project's database.
  \item Everything else, including tentative or ephemeral records, planning or strategy documents, and any representation that includes the perspectives of the people who created the record in a prominent role, are either dumped on a hard drive, discarded, or forgotten.
\end{enumerate}

\section{Photography}
\begin{enumerate}
  \item This is apparent in how different kinds of photos are valued.
  \item Archaeological photographs serve as records that stand in for a physical object or set of material configurations that are generally not accessible, either because they are destroyed through excavation processes or because of logistical constraints that make it difficult to directly engage with the physical object or collection.
  \item This is evident through the fact that much of a photograph’s significance is drawn from associations made between a photo and records made about the photo.
  \item More specifically, each photo is recorded in a ``photo log'', or a spreadsheet where the objects being photographed are identified and the circumstances under which the photos were taken are documented.
  \item This is actually what distinguishes an official or authoritative photo from a casual photo, or photos of people at work.
  \item These photos, which actually show people, including people engaging with the photographer, e.g. by smiling or making funny faces, are dumped on a shared drive for personal recollection or to serve as very imprecise memnonic records that jog a viewer's memory of the circumstances under which work was performed.
  \item Offial photography, on the other hand, is now understood as a form of data collection, and as such must be performed with consistency and in a relatively objective manner.
  \item When not performed by a dedicated photographer, photography in fieldwork settings is often performed in a manner that is directed by data management specialists.
  \item Fieldworkers are instructed to capture a specific set of images when triggered by certain events, i.e. when opening and closing a trench or after identifying a significant archaeological feature.
  \item They are required to capture ``complete sets'' of photos, i.e. photos of a trench opening and closing, with and without a photo board for each instance.
  \item And fieldworkers are instructed how to optimize the outputs of their efforts, such as by managing lighting and eliminating problematic shadows.
  \item However, the significance and value of this work is rarely made clear to fieldworkers who actually take the photographs.
  \item Creative agency is relocated to those occupying managerial roles, who are responsible for designing photography protocols, selecting and configuring equipment, and establishing how relevant information about a photograph should be processed.
  \item Archaeological photography is thus made into one aspect of formulaic data collection procedures.
\end{enumerate}

\section{Data Management and Situated Representations}
\begin{enumerate}
  \item This leads me to one of my main arguments, that it is impossible to distinguish between the management of data and the management of labour that produced them.
  \item This is despite the tendency for databases to completely ignore the ways in which data were produced, as I described a moment ago.
  \item This built-in ignorance of the mechanisms through which data arrived at their present state contributes to a disconnect between representation and reality.
  \item This is exemplified in two examples, namely the value placed on field journals and site tours, two very situated and informal modes of archaeological representation, as means of legitimizing more formal records.
\end{enumerate}

\section{Field Journals}
\begin{enumerate}
  \item Field journals describe the gradual act of discovery whereby the trench and its constituent excavation units gain significance in relation to the project's overall aims.
  \item Journal entries switch between atomic and descriptive characterizations of specific elements within the trench and more speculative associations that draw the trench within a broader understanding of the site as a whole.
  \item In this way, field journals are discursive media in that they describe and discuss particular aspects of a project from situated perspectives, and contextualize and define an object's significance based on particular experiences with it.
  \item In the cases I observed, field journals were never transcribed or codified into formal representations, or generally applied towards analytical tasks.
  \item They are used as memnotic devices, to recall and reconstruct the mindset of the person who worked with the recorded material at specific moments in time.
  \item This was exemplified by Liz's experiences collecting and integrating data from projects that she has not participated in.
  \item she recalls that ``the very first thing I do is read the notebooks, because that’s where you get sort of what their, \dots their sort of thought process throughout the whole thing, which makes it so much easier to then dive into the actual meticulous data''.
  \item Reading field journals draws Liz into the project, enabling her to share in the experiences of archaeological discovery, thus establishing a kind of parasocial relationship that affords her with a similar degree of understanding as an actual team participant might have had.
  \item Obtaining access to field journal, thus, understandably, requires that trust be established.
  \item Liz had to reach out to project directors to request access.
  \item She had to make her case with a value proposition, and enter into a collaborative commitment in which she was placed in a dependent position.
  
  \item Barry also identified professional tensions that emerged upon the death of the director of a historically impactful archaeological project that served as a venue where many very well known archaeologists were trained during their formative years.
  \item The late director's successor mandated that all field journals be made openly accessible alongside all other records, in compliance with still-emerging open data standards and expectations.
  \item Many former participants objected to this, and the issue was nearly even litigated upon.
  \item Barry never stated a specific reason for this contention other than the general sense that publishing these old records broke an unspoken professional courtesy, and that a trusting relationship between archaeologists and their late mentor had been impeded upon.
  \item A collaborative commitment had been forged, was broken, and then suddenly restructured in a manner that was contradictory to the original arrangement.
\end{enumerate}

\section{Site Tours}
\begin{enumerate}
  \item The written accounts of archaeological discovery recorded in field journals are sometimes elaborated upon during site tours, which are either regularly scheduled or sporadic events whereby the whole team goes around the site to better understand what is going on in each trench.
  \item When arriving at a trench, its supervisor describes the trench, typically in a fashion that recalls the work and decisions involved in its creation.
  \item This is usually supplemented by interjections or rebuttals made by the project director or finds analysts who situate the trench in relation to broader project-wide narratives.
  \item Site tours are informal and are never recorded, but convey a great deal of information to listeners, and I thus consider them to be a form of data sharing.
  \item They typically use imprecise language and refer to things whose meanings may not be well understood outside the project team.
  \item For instance, members of Case A often refer to the ``red shit,'' which signifies a layer of red clay that appears throughout the site, and which nearly all excavators have had to struggle with.
  \item Project directors like to give these tours to visiting scholars, notable guests, and new project participants so that they can get a better understanding of what is going on in the site, rather than being limited to what's published in a paper or report.
  \item In a particularly noteworthy instance, Basil was eager for a visiting consultant to see the site and the material first hand, since the unique conditions under which the archaeological materials were preserved made them appear very weathered, and it was difficult to record them in a manner that convinces a reader about their specific characteristics.
  \item Basil invited an esteemed colleague to come and see the materials first hand, and to gain her confidence.
  \item He was hoping to translate this informal experience into a ``personal communication'' citation, which, while not normally considered to be a very strong way of backing an argument, would have made a significant difference in this case.
  \item The visiting colleague is generally regarded as the primary authority on this particular kind of material in the region, and her notable experience would carry weight via word-of-mouth communication, especially with regards to such weathered material preserved in contexts whose stratigraphy can only be dated with coarse resolution using absolute dating methods.
  \item As a consulting specialist, the visitor was able to informally commit her authoritative reputation to the project, which granted legitimacy to the formal records whose validity could not necessarily be held up on their own merit despite being accurate representations of significant archaeological phenomena.
\end{enumerate}

\section{summary of these tensions}
\begin{enumerate}
  \item Field journals and site tours contain very rich information about physical and social circumstances regarding data’s creation.
  \item They are communicated in trust, and only accessible by understanding colleagues.
  \item Despite including information that is critical for making sense of formal records in contexts of reuse, is it awkward to translate them into schema that can be leveraged in formal representations.
\end{enumerate}

\section{Tensions with Open Data Infrastrcutures}
\begin{enumerate}
  \item Finally, I'll elaborate briefly on how this relates to open data infrastructures.
  \item Most open data platforms seem to identify projects as the fundamental units of concern.
  \item However, they tend to hide or ignore the boundaries and commitments that exist within projects.
  \item Power is concentrated in the directors' position, who serves as the primary contact, and as the recognized owner and controller of the totality of a project's information.
  \item The roles of other stakeholders are ignored or diminished, even though they may have been more familiar with the circumstances of the data's creation and value.
  \item Datasets are presented as they are in their final state, and there is no way to verify the accuracy of descriptive metadata or paradata when reading these records at face value.
  \item Numerous studies have documented how archaeologists work around these platforms to obtain additional contextual information.
  \item I interpret these actions as archaeologists seeking out a relationship that more closely resembles a collegial relationship, to gain insight into things that archaeologists are not able or not willing to publicly share.
  \item In other words, archaeologists on either end of the archive are not fully satisfied with the unidimensional upload/download mechanic of data sharing.
  \item I find it quite ironic that the formal and transactional protocols meant to streamline mutual comprehension of a dataset reveal their own inadequacy for achieving their stated purpose, and the strengths of the system that they are meant to replace.
\end{enumerate}


\section{Where do we go from here?}
\begin{enumerate}
  \item Does this mean that these systems are useless?
  \item Absolutely not!
  \item But perhaps we can improve them to account for, or help introduce, collaborative commitments that are more amenable to archaeologists.
  \item Perhaps this calls for a paradigm shift, which emphasizes data as the product of pragmatic and circumstantial decisions, and less stable products.
  \item This might involve adopting a registry rather than repository mentality, whereby data presented in association with the people who produced them, and the value that the data held to achieve their specific goals.
  \item And although there may be pushback, I think we should also begin to ensure that depositors maintain some respobsibility over their data that they produce and publish.
  \item In making data available on the web, depositors are establishing a commitment with anyone who wants to reuse the data.
  \item But this commitment is easy to reneg on, and it is easy to ignore any emails that come in asking for support.
  \item So long-term commitment to what we put out there might help facilitate reuse (and imbue the data with greater functional value as a result), or otherwise encourage depositors to write better documentation.
  
  \item Registry rather than repository mentality
  \item Provide the means to establish meaningful collaborative ties
  \item Recognize the value behind various kinds of contributions
  \item Accept and acknowledge friction
  \item Embrace anarchy

  \item Accept and acknowledge the complicated imperfections, like fuzzy notions of authorship and spotty inclusion of informal records in digital repositories
\end{enumerate}

\subsection{Data Management Systems}
\begin{enumerate}
  \item It is also noteworthy that only activities where upper management wields creative agency hold the potential to garner some legitimately valuable outcome.
  \item For instance, excavators produce records that are not useful in themselves, and the products of their work must be integrated and analyzed by publishing members of the team in order to yield valued findings.
  \item Once the information collected in the trench leaves that setting and enters the dig house or the database, it enters a different functional realm, no longer a product derived from intimate familiarity with a trench, and becomes a generic resource to be leveraged by a publishing member of the team.
\end{enumerate}


\section{Questions?}
\end{document}
